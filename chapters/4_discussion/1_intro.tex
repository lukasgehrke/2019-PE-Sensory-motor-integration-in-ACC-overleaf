\section{Discussion @ALL}

% - discuss with spatial conflict processing: Savoie, & Simon Task results: cohen, cavanagh, toellner
% - reference to self and body ownership, spatial computations between egocentric and allocentric? cite Ehrsson, Slater, Gonzales-Franco (uncanny valley of haptics)
% - usage in online BCI applications detecting glitches etc., see CHI work, Zander, Blankertz
% - single-trial regression challenges in mobile brain body imaging studies: energy of stimulus, Time-locked vs. continuous regressors/stimuli, EEG artifacts due to movement, Contrast desktop stimulation vs. wide FOV in VR, Higher cortical noise, Saliency of stimulus, necessary attention on stimulus %in terms of predictive coding,
% - Study specific shortcomings: low N, both in subjects and in trials due to oddball paradigm


%%%% writing ressources below
% - discuss reasons for low R^2 and approaches to address it (experiment design and methods (unfolding other processes, adding regressors etc.)) is there a meta study on effect sizes in timefrequency resolved EEG studies. What does Johanna report, regarding effect size?
% - is proprioceptive feedback important in our task: I'd argue against that because I consider vibration on the fingertip to be an exteroceptive sensation and do not consider it relevant where the hand was and how my joints were angled etc. during the mismatch event, therefore i stick to exteroceptive prediction errors only!
% - On world stability: designing visuo-haptic mismatches in virtual environments means further complicating the action-perception cycle understood as hypothesis testing. By pseudo-randomizing when one of the 25 percent chance oddballs appears, hypothesis testing and subsequent evaluation of the latent variables causing the oddball to appear becomes meaningless.