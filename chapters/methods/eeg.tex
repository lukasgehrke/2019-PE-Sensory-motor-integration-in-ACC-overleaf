\subsection{EEG}
160 channel EEG data sampled at 1000 Hz were recorded wireless using the BrainProducts MOVE System (Brain Products GmbH, Gilching, Germany) 128 channels applied on the head and 32 channels on the neck.

% start todo clean up
% how many channels were removed
% what where the ICLabel settings for cleaning?
\subsubsection{Single-subject EEG data processing}
Artifactual channels were manually identified, removed, and interpolated using spherical interpolation (on average, X channels were interpolated, SD=Y). Subsequently, the data was re-referenced to the average of all channels and a zero-phase Hamming windowed high-pass FIR filter (order 827, pass-band edge 1 Hz) was applied to the data. Concatenated data from all individual explorations was parsed into maximally independent components (IC) using the adaptive mixture of independent component analyzers (AMICA) algorithm with prior principal component analysis (PCA) reduction to the remaining rank after interpolation\citep{Palmer2011}. Subsequently, independent components reflecting eye activity were subtracted from the data in order for an automated cleaning procedure across channels in the time domain not to be affected by eye signals. All Components were automatically matched against the ‘ICLabel’ database and components with a probability >= .XY to reflect eye activity were subtracted\citep{iclabel}. Subsequently, 15\% of the noisiest time segments were determined by jointly ranking the amplitude mean, standard deviation, and mahalanobis distance across channels of consecutive one-second epochs\citep{cleaning_fh2018}. Subsequently, artifactual time windows were rejected from the concatenated data prior to eye component removal and a second ICA was computed on this cleaned data containing eye movement signals. Hence in a first step eye movement components were rejected to subsequently find noisy segments in the data not related to eye movements.
For each IC, an equivalent dipole model was computed as implemented by DIPFIT routines in EEGLAB. For this purpose the individually measured electrode locations were rotated and scaled to fit a boundary element head model (BEM) based on the MNI brain (Montreal Neurological Institute, MNI, Montreal, QC, Canada). We refer to the approximated spatial origin of an IC as “in or near” a specified location.
Epochs of three second length were created around the wall touch event. A spectrogram of all single trials was computed for all IC activation time courses using the newtimef() function of EEGLAB (3 to 100 Hz in logarithmic scale, using a wavelet transformation with 3 cycles for the lowest frequency and a linear increase with frequency of 0.5 cycles).
Due to the interactivity of the wall touch events, all touches were normalized in length using a linear timewarp. Single trial power values were warped to the median length across all wall touches (~714ms).

%the 70 ICs of each participant explaining most of the variance of the data were selected (1330 ICs in total) 
% TODO
% ICLabel settings selecting components
\subsubsection{Group-level analyses: Clustering}
To allow for group-level comparisons of EEG data at the source level (ICs), components with a probability of XY using ICLabel were selected and subsequently clustered based on their equivalent dipole locations (weight=6), grand-average ERSPs (weight=3), mean log spectra (weight=1), and scalp topography (weight=1), using a region of interest (ROI) driven repetitive k-means clustering approach\citep{cleaning_FH2018}. The weighted IC measures were summed and compressed using PCA, resulting in a 10-dimensional feature vector for clustering. ICs were clustered by applying the k-means algorithm with k equaling 70\% of the mean number of ICs retained across subjects. We chose to use fewer clusters than ICs per participant because of our assumption that, although statistically independent per time point, there may be more than one IC per participant that is similar in function and location. ICs with a distance of more than three standard deviations from any final centroid mean were considered outliers.

% todo: update with final clustering solution properties
To ensure replicability of the clustering, we employed the same clustering approach reported in\cite{cleaning_FH2018}. After applying desirable weights (number of participants: 3, ICs/participants: -1, spread: -1, RV: -1, distance from ROI: -2, Mahalanobis distance from the median: -1) the final clustering solution contained the ICs from 25 participants, a ratio of 1.13 ICs per participant (two participants with two ICs each), a spread of 296, a mean RV of 10.3\%, and a distance of 10.2 units in the Talairach space for the cluster of primary interest.

%averaging of ICs → citation why that is either okay (Delorme?) or problematic :), one reason why doing this makes sense is to avoid higher weighting of individual subjects based on more ICs per cluster.
\subsubsection{Group-level analyses: Statistical parametric spectra and time-frequency maps}
Spectra and event-related spectral perturbations (ERSP\citep{Makeig2004}) were analyzed on the group level using robust repeated-measures ANOVA\citep{Pernet2011} to test for the effect of repetition and multiple regressions to assess the influence of each behavioral parameter individually. Initially, we obtained estimates of each subjects response by mass-univariate modeling of each ERSP time-frequency point. We fitted the model $Y = Maze*Repetition + Touch_Duration$ to control for the effect of touch duration. Since allowing continuous interaction with the touchable walls introduced the confound of touches/epochs of different duration. The resulting regression estimates (Betas) were used as each single-subjects summary for group-level inferences.

To characterize the mapping between event-related brain dynamics and the stimulus properties of the time-locked events we employed a hierarchical linear modeling approach on the time-frequency data. First, single-trial regressions weights were estimated for the factors maze and trial run, their interaction, as well as a continuous noise predictor (touch duration “slope rERSP”) for each time-frequency pixel across trials [Rousselet 2008,2009,2010]. We chose a resolution of 96 frequencies  (~3 to ~100 Hz) and 177 timepoints (-300ms to 714ms) around the wall touch event. Since single trials were warped linearly to the same length we corrected for the warping by regressing the touch duration (see above) on each time-frequency pixel across single trials. Before regression, touch duration was centered on 0. Hence, estimates for the categorical predictors are interpreted when touch duration is at its mean value. After adding back the intercept and summing across the corresponding regression terms for each of the 12 factor levels (3 Maze Trials X 4 Mazes), single trial estimates were averaged (in power) for each factor level. Whole epoch average power per factor level was then baseline corrected using a divisive baseline, i.e. mean power values in each frequency across the -300 to -100ms window pre wall touch. Subsequently, power values were transformed to the decibel scale using 10 * log10 (factor level average baseline corrected power).

Using LIMO EEG, subject estimates from the first level were taken as input for group level inference testing [Pernet2011,Rousselet2011,Cohen2011]. To investigate the main effect of both factors, run and maze, we computed 3 (Runs) X 4 (Maze) repeated-measures ANOVAs for each time-frequency pixel for each IC k-means cluster independently. We addressed the multiple comparison problem by correcting the FWER using the threshold-free cluster-enhancement statistic (TFCE, [cite smith and nicols, 2014]). TFCE statistical maps were computed for each of 1000 bootstraps obtained by sampling with replacement across participants. A distribution of maximum TFCE scores was build, keeping the max TFCE score per bootstrapped time-frequency map. The TFCE scores of the true result were then thresholded with the 0.95 percentile (alpha = 0.05) score of the max TFCE distribution.

In order to investigate the relationship between ERSPs and parameters of spatial learning, we build a regression model using group level data as input. Taking factor level data as the dependent variable, we entered both sketchmap scores and exploration duration as continuous predictors, assuming a continuum of a mental spatial representation represented by discrete sketchmap scores. For simplicity, we did not consider any interaction terms with the experimental design (3 maze trials X 4 maze). As above, to correct for multiple comparisons, TFCE scores of bootstrapped samples were used to build a max TFCE distribution to determine the alpha level threshold value.

For visualization purposes, group average ERSPs were obtained by averaging dB time-frequency maps significance masked by threshold surviving TFCE scores.
% end clean up