\subsection{Task and study design}
Participants explored four invisible mazes of increasing complexity three times in a row. The order of the mazes was identical across participants with an increase in difficulty across consecutive mazes by introducing an additional turn or changing turn direction. For each maze, participants were tasked to find the end of the maze (dead-end) and subsequently find their way back to the start. The main effects of the factors complexity and repetition were analyzed by fitting three individual linear models and computing ANCOVAs of (a) sketch maps assessing the usefulness of the mental spatial representation for navigational ends, see section 2.6 "Sketch Map Measure of Spatial Ability"\citep{Gehrke2018}, (b) the time-on-task and (c) the number of wall touches per maze and repetition.
We expected participants to be faster and require fewer number of touches for each repetition within the same maze and further hypothesized a shift towards increasing usefulness of the sketch maps. For a rich description of task, procedure and hardware setup as well as behavioral parameters and administered psychometric and self-report measures, please consult section 2 of\citep{Gehrke2018}. Further, the same data was rejected as described in section 2.6 "Trial Rejection" in\citep{Gehrke2018}.