\subsection{Summary and hypotheses \textcolor{green}{Lukas}}
% todo Lukas: write a comment on what should be here

In this work we adhere to the following theme. Following our previous findings, we first investigate whether event-related potentials at select channels correspond to the prediction error framework using more data compared to our previous analyses. Then we examine both, the impact of haptic feedback as well as the impact of the active sampling behavior, i.e. hand movement velocity, on the ensuing prediction error by means of single-trial regression. Subsequently, we assess whether the subjective presence experience moderates the impact of haptics, velocity and their interaction on the ERP. Next, we extend our analyses to the time-frequency domain on the source level of independent components. We cluster independent components for group level inference and conduct the identical analyses as on the channel level for one select region of interest, assigned to the anterior cingulate cortex. Importantly, we hope to demonstrate the application of single-trial multiple regression, robust multilevel modeling correcting for multiple comparisons using cluster-based correction on the group level for the applied mobile brain/body imaging (MoBI) researcher.

1st: We use the weights returned through LDA classification between match and mismatch trials to locate our signal of interest. Classification, based on ERPs, strongly relied on the activity of independent sources in posterior cingulate cortex. We chose this location as a seed to find and subsequently optimize a cluster of independent components. For this cluster we computed mass-univariate single-trial regressions on ERPs as well as ERSP (event-related spectral perturbations) to map ongoing motor behavior, i.e. hand velocity, and feedback "quality" to posterior cingulate source dynamics during perceived mismatch. Ultimately, we link mismatch processing source signals to post-error slowing situating our findings within reward prediction error theories of reinforcement learning. In summary, our findings indicate the role of posterior cingulate theta activity in sensory-motor coupling perhaps eliciting an error signal, propagating needed adaptions to a changing world throughout the network.

% We close with a discussion on future challenges, specifically regarding stimulus energy