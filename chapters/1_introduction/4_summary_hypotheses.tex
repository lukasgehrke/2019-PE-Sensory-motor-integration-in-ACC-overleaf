\subsection{Summary and hypotheses}
In this work we investigated the impact of \textit{perceptual context} on EEG source signals promptly following a perturbation in the spatio-temporal task structure. To this end, we followed a data-driven approach: (1) we located independent brain sources that maximally contributed to a single-trial linear classification between predicted and perturbed sensory outcomes in an object selection task; (2), then, we contextualised sensory prediction errors at the moment they occurred by participants' instantaneous hand velocity and the haptic realism of the environment; (3) then, by means of a single-trial regression we described the contextual impact on sensory prediction errors elicited at sources maximally contributing to the binary classification; as a result, (4) we showed how behavioral adaptation via post-error slowing, situating our findings within reward prediction error theories of reinforcement learning. In summary, our findings indicate a role of parietal activity in spatio-temporal binding.


%%%% writing resources below:
%perhaps eliciting an error signal, propagating needed adaptions to a changing world throughout the network.

%Following our previous findings, we first investigated whether event-related potentials at select channels demonstrate activity patterns that correspond to the prediction error framework using more data compared to our previous analyses. We then examined both, the impact of haptic feedback as well as the impact of the active sampling behavior, i.e. hand movement velocity, on the ensuing prediction error by means of single-trial regression. Subsequently, we assess whether the subjective presence experience moderates the impact of haptics, velocity and their interaction on the ERP. Next, we extend our analyses to the time-frequency domain on the source level of independent components. To this end, we clusterd independent components for group level inference and conduct the identical analyses as on the channel level for one select region of interest, assigned to the anterior cingulate cortex. Importantly, we demonstrate the application of single-trial multiple regression, robust multilevel modeling correcting for multiple comparisons using cluster-based correction on the group level for the applied mobile brain/body imaging (MoBI) researcher.

%First we use the weights returned through LDA classification between synchronous and asynchronous trials to locate our signal of interest. Classification, based on ERPs, strongly relied on the activity of independent sources located to parietal cortex. We chose this location as a seed to find and subsequently optimize a cluster of independent components. For this cluster we computed mass-univariate single-trial regressions on ERPs as well as ERSP (event-related spectral perturbations) to map contextual parameters, i.e. hand velocity, and feedback "quality" to parietal source dynamics during spatio-temporal prediction errors.