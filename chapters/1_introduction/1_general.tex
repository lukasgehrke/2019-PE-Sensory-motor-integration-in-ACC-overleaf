\section{Introduction (Lukas, Sezen, Klaus, Pedro)}
In order for brains to accurately infer about the latent variables governing the environment, for example gravity, recent theories propose that brains set out to actively sample their environment to confirm predictions of prior hypotheses. \cites{Clark2013, Friston2010, Rao1999}. This active sampling, also labelled active inference, is inherently tied to the bodies capabilities to act on the world, rendering the action-perception cycle of cognition a deeply embodied process \cite{Friston2012}. The minimisation of surprise or prediction error, as the incentive of active inference, has been extensively studied in relation to perception in motor control. Here, active inference typically considers engaged viewing as reflected in saccadic eye movements as an observable manifestation of how the brain directs motor output during active sampling of the environment. However, environmental affordances surpass the visual domain. Particularly in humans and apes, a proclivity to use both hands to act on the world emerged and are greatly trusted upon. So much so, that many tasks can be completed without concurrently consulting the visual domain, like typewriting or even reaching for a cup. On the computational level, a "bidirectional cascade of cortical processing" arguably underlies such inferential processing \cite{Clark2013}. A hierarchical generative model, or forward model, of the environment is assumed that is trained to minimize prediction errors from elicited actions, or in other words "actively constructing explanations" \cite{Wolpert2011, Friston2018}. 

Mobile Brain/Body Imaging (MoBI; \cites{Gramann2011a, Gramann2014, Makeig2009} provides a well situated approach to assess both, the physical realisation propagating prediction errors in response to all afforded actions as well as the behavior preceding the currently propagated sample and ultimately, the ensuing behavioral adaptation. For example, in order to assess the the brain dynamics as implementation of predicitve porcessing, EEG responses can be recorded and linked back to the behavioral level. This allows for observing how the brain organizes behavior, including motor planning and control. Several developments have concluded the statistical foundation to address the many-to-many mapping challenge in cognitive neuroscience through single-trials analysis. The results provide the tools to directly map from body, or behavior space to brain space \cites{Pernet2011, Bridwell2018, Friston1994a, Blankertz2011}. In other words, sampling the context in which prediction errors occur during action-perception and then directly map the instantaneous context to the ensuing brain response can be employed to investigate the bidirectional evidence-integration being computed, i.e. action-oriented predictive processing \cite{Clark2013}.