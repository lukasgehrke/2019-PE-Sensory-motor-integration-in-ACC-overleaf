\section{Introduction}
In order for brains to accurately infer about the latent variables governing the behavior of the environment, e.g. gravity, they set out to actively sample their environment and confirm prior predictions \cites{friston, gregory1989}. This active sampling is inherently tied to the bodies capabilities to act on the world, rendering the action-perception cycle of cognition a deeply embodied process \cite{Friston2012}. The unifying brain theory that is the free-energy principle and its consequence active inference pose extraordinary opportunities for mobile brain body imaging \cites{gramann, makeig}. The minimisation of surprise or prediction error, as the incentive of active inference has been extensively studied in relation to perception, i.e. in terms of exteroceptive prediction errors \cites{cohen, cavahang, mmn_stuff}. Here, active inference typically considers engaged viewing, saccadic eye movements, as an observable manifestation of how the brain directs motor output during active sampling of the environment. However, environmental affordances surpass the visual domain. Particularly in humans and apes, a proclivity to use both hands to act on the world emerged and are greatly trusted upon. So much so, that many tasks can be completed without consulting the visual domain at all, e.g. typewriting or even reaching for a cup. 

Understanding action-perception as hypothesis testing, or active inference, means sampling the environment through targeted reaches in order to maximize evidence about the consequences of the elicited actions \cite{gregory1980}. The internal model of the world is trained to maximize predictive accuracy of the consequences of actions, or in other words actively constructing explanations \cite{friston2018}.

Here, we investigate exteroceptive prediction errors during reaching movements similar to reaching for a cup. Simultaneous mobile brain and body imaging provides a well situated approach to assess both, the physical realisation propagating exteroceptive prediction errors in response to all afforded actions as well as the behavior preceding the currently propagated sample and ultimately, the ensuing behavioral adaptation. For example, in order to assess the implementation, or brain dynamics, we can measure EEG responses and link them back to the behavioral level, observing how the brain organizes its behavior, i.e. motor planning and control. Recently, several developments have concluded the statistical foundation to address the many-to-many mapping challenge, providing the tools to directly map from body, or behavior space to brain space \cites{pernet2011, cohen2011, cavanagh2018, pernet, friston, cavanagh, calhoun, ehinger, dimigen, kutas, groppe}. Sampling the context in which exteroceptive prediction errors occur during action-perception and then directly map the instantaneous context to the ensuing brain response closely assesses the evidence-integration being computed, i.e. predictive coding.

% teaser and general intro: the brain as a predictive mechanism, embodiment, active inference
% address the many-to-many mapping problem: single-trial regression to map situational context during action-perception to brain activity.

% add intro about why single-trial analysis in mobi is a good idea, then go over to the example at hand