\section{Introduction}
Recent theories propose that brains actively sample their environment to confirm predictions of prior hypotheses~\cites{Clark2013, Friston2010, Rao1999}; this view might explain how a subject accurately infer what variables govern action in their environment, for example the effects of gravity. This active sampling and inference is inherently tied to the body's capacity to act on the environment, rendering the action-perception cycle of cognition into a deeply embodied process~\cite{Friston2012}. The minimisation of surprise, or prediction error, as the objective function of active inference, has been extensively studied with regards to perception in motor control. Here, active inference typically considers engaged viewing as reflected in saccadic eye movements as an observable manifestation of how the brain directs motor output during active sampling of the environment. However, environmental affordances surpass the visual domain. Particularly in humans and apes, a proclivity to use both hands to act on the environment has emerged and is greatly trusted upon. In fact, many tasks can be completed without concurrently consulting the visual domain, such as typewriting or even reaching for a cup; these typically rely heavily on the tactile and proprioceptive sense and as such are denoted as eyes-free interactions. On the modelling level, a "bidirectional cascade of cortical processing" arguably underlies such inferential processing \cite{Clark2013}. A hierarchical generative model of the environment, or forward model, is assumed that is trained to minimize prediction errors from elicited actions that test this forward model, or in other words "actively constructing explanations" \cites{Wolpert2011, Friston2018}.

The underlying assumption of most contemporary motor control theories is that a generative model explains potential variability between elicited motor commands, i.e., the movement they trigger, and changes in body pose and the environment \cites{Wolpert2011, Shadmehr2010}. Assuming the brain entertains models of its environment, such a forward model that predicts sensory consequences of an action given the current context and an efference copy of the motor command, meanwhile refining the models of the causes of those sensory consequences \cites{Pearson2011, Friston2010, Friston2016a}. Evidence about the exact nature of the computational level in the brain is converging on hierarchical Bayesian models computing inference of predictions from a generative model, or what is predicted to happen, taking into account sensory \textit{evidence}, or what is currently perceived as an action outcome \cite{Knill2004, Shams2010}. Taken together, model evidence is scored by an accuracy or surprise metric, for example, prediction errors. Importantly, in order to make predictions in a stable environment, where inference on the latent variables governing sensory events is useful, the brain must be a good model of this environment, grounded in causality \cite{Friston2016a}.

%high level comment by pedro


%%%%% writing ressources
%Mobile Brain/Body Imaging (MoBI; \cites{Gramann2011a, Gramann2014, Makeig2009} provides a well situated approach to assess both, the physical realisation propagating prediction errors in response to all afforded actions as well as the behavior preceding the currently propagated sample and ultimately, the ensuing behavioral adaptation. This allows for observing how the brain organizes behavior, including motor planning and control and how prediction errors resulting from actions are propagated back to inform the generative model. Sampling the context in which prediction errors occur during action-perception and directly map these to the subject's real-time context allows to understand \textit{natural} movement behavior such as reaching tasks, much in the vein of the aforementioned eyes-free tasks. As such, a direct mapping from behavior space to brain space converges to the bidirectional evidence-integration being computed, i.e. action-oriented predictive processing \cite{Clark2013}. This however, requires a statistical foundation to address the many-to-many mapping challenge in cognitive neuroscience through single-trials analysis \cites{Pernet2011, Bridwell2018, Friston1994a, Blankertz2011}.