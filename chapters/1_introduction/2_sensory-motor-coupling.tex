\subsection{Visuomotor adaptation during targeted reaching \textcolor{green}{Sezen, Klaus, Pedro}}
Previously, we reported results in light of applied human-computer interaction scenarios and which achieving haptic realism is one of the key challenges, specifically in virtual reality development. Here, the goal of haptic devices is to render realistic sensory feedback that mimics the sensory experience a user would normally perceive when acting on the real world. Head-mounted virtual reality in combination with expected haptic feedback renders a \textit{naturalistic} experience. Further, we presume that predicting the hidden causes of the observed outcomes within the action-perception cycle should, hypothetically, depend on how present one feels in the environment. Virtual reality can provide experiences that elicit profound presence experiences \cite{}. We introduced a paradigm to assess whether prediction error-related potentials at fronto-central electrodes can, on average, detect an increase in the level of immersion, operationalized through added feedback channels \cite{Gehrke_2019}. In the task, we render a perturbation in a fraction of the trials, distorting predicted visual or visuo-haptic consequences of the motor command \cite{oddball}. Specifically, we perturb visuo-haptic feedback conflicting forward model predictions about when or where the current hand position elicits object selection. In the current work we investigate the impact of \textit{perceptual context} on EEG source signals promptly following visuomotor adaptation. We follow a data-driven approach: first we locate independent brain sources maximally contributing to a single-trial linear classification between predicted and perturbed sensory outcomes in an object selection task. Then we contextualise sensory prediction errors at the moment they occur by participants instantaneous hand velocity and the haptic realism of the environment. Using single-trial regression we describe the contextual impact on sensory prediction errors elicited at sources maximally contributing to the binary classification.

% in the following:
% - first describe general ideas of visuomotor adaption and specify prediction error in our task and differentiate from other prediction errors, be precise!
% - then introduce other EEG findings using similar task focusing on spatial conflict processing

% describe previous findings
% - discrimation classes of prediction error (Wolpert), sensory prediction errors (shadmehr), visuomotor adaptation (shadmehr); is proprioceptive feedback important in our task: I'd argue against that because I consider vibration on the fingertip to be an exteroceptive sensation and do not consider it relevant where the hand was and how my joints were angled etc. during the mismatch event, therefore i stick to exteroceptive prediction errors only!
% - cerebellar (shadmehr, krakauer)
% - parietal (Savoie); posterior parietal sensory motor integration within the prediction error hierarchy
% - acc/PEN/FRN/ERN (Holroyd, Cohen, Zander, Akman): we dont hypothesize ACC since we do not provide a reward in the task. There is no evaluation with respect to a task goal