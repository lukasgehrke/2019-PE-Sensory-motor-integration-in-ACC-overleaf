\subsection{Processing spatio-temporal asynchrony}
Through evolution, the human sensory organs have developed to actively sample the environment with regards to sourcing food and reproduction. Individual senses each excel in specific sensory domains with a combination of reliability-weighted sensory cues providing a holistic sample \cite{Fetsch2012, Cao2019}. Such samples gain their perceptual "significance through their meaning about their causes" \cite{Kording2007}. Thus, in order to efficiently infer what causes certain sensory events, a multisensory perceptual experience is preferable with temporal correlations and spatio-temporal congruency serving as causal binding cues \cite{Robertson2003}. A modern day example of a "laggy" internet connection in which commands typed on a keyboard are rendered with a delay, exhibits a violation of spatio-temporal congruency. In this case an undesirable outcome. The recent advent of immersive technologies has emphasized the importance of congruent sensory channels. Simulating holistic sensory samples hinges on the alignment between rendered channels. Frequently misaligned cues, for example visuo-haptic "glitches", impact the overall user experience. Previously, we explored how these visuo-haptic discrepancies affect at  reaching task with both visual and haptic feedback in a virtual reality environment (VR)\cite{Gehrke2019}. In our paradigm, a 25\% of the total trials featured an (unknown to the participant) asynchronous spatio-temporal cue by prematurely triggering the object selection as participants reached out to grab the virtual object. In our task, successful grasp of the virtual object was indicated by a color-change of the reached-to object accompanied by a synchronous vibration under the fingertip hinting at haptic rigidity. By maintaining spatio-temporal synchronization in the remainder 75\% of the trials, we hypothesized the formation of a predictive model on the causes of sensory events, therefore amplifying a spatio-temporal prediction error in the asynchronous trials. By investigating participants in this unconstrained reaching setting, we extended classical setups enabling ongoing proprioceptive and vestibular sensing while actively sampling peripersonal space. Our motivation was to investigate how the momentary sensory context influences processing of spatio-temporal asynchrony eliciting prediction errors on the latent causal structure of the environment. Furthermore, current modeling work argues that sense of agency, believing that ones actions were the cause of an event, emerges when the outcome caused by an action is in line with prior predictions on the latent variables governing the environment and the objects therein \cite{Moore2016, Legaspi2019}. In other words, sense of agency emerges from an understanding how "I" (i.e., the subject), can actively influence the environment to reveal the causes governing its behavior, for example letting go of an apple to the floor to infer gravity's effects on it.

On a neuroanatomical level, recent evidence points towards a parietal-frontal hierarchy binding sensory signals to a common source while segregating those from diverging and guiding future adaptation. Within this hierarchical causal inference, evidence abounds that primary sensory areas operate under the assumption of separate signal sources whereas parietal areas appear to compute under the assumption of common causes with precision-weighted unisensory inputs. Further, anterior parietal areas then add a reliability-weighting term to the holistic causal structure \cite{Cao2019, Rohe2019, Cohen2013a}. Ultimately, adaptation of future behavior is thought to be driven by parietal and/or frontal areas. Presumably depending on the broader context, for example valence assessments, behavioral adaptation and policy selection invoke frontal or parietal areas \cite{Pearson2011, Kolling2016, Holroyd2002}. Employing a variation of the classical oddball paradigm, our work emphasizes signals operating under the assumption of a causal fusion model.

Prominently, theta- and alpha-band activity, as well as their finer granularities, have been implicated in cognitive processing. \citet{VanRullen2016} argued for the notion of \textit{perceptual cycles} alluding to these frequencies as the impulse generators of perception. \citet{Rohe2019} implicate prestimulus alpha frequencies in \textit{rhythmic perception} modulating the tendency whether following spatio-temporal binding challenges are more likely to resolve in binding or segregation. Parietal areas, as a location of arising alpha rhythms have been frequently linked to anchoring the bodily self during action-perception being ideally located for integration of spatial reference frames \cite{Blanke2015, Guterstam2015a, Halgren2019}. Further, connections to the hippocampus via retrosplenial cortices implicate a role in spatio-temporal context situating and reproduction \cite{Pearson2011, Clark2018, Gramann2009, Friston2016a}. The extensive research on theta frequencies, with a frequent localized to frontal areas, have concluded a role in cognitive control \cite{Cavanagh2014}. Recently, \citet{Duprez2020} have demonstrated theta frequencies as a messenger directing behavioral adaptation through cognitive control. Following a cognitive demand for adaption, theta motivates rich behavioral policy changes exceeding simple reaction time measures \cite{Cooper2019}. 



%%%% writing resources below:
%This characteristic can be employed for brain-computer interface purposes. Information extracted via \textit{cognitive probing} can be employed to inform technical systems of the brains intent \cite{Krol2020a, Zander2016}.










% situate task in general on three timescales
%In order to situate our report within the literature, we first consider how instantaneous prediction errors are influenced at different time and contextual scales. 
% "spanning from proprioceptive to visual cues"

%In the long run, the brain is in the game of optimizing the outcome of its motor behaviors. On any given motor task, learning is ultimately driven by optimizing a trade-off between domain-specific and generalizable motor skill.

% triggering prediction errors at different time and contextual scales.
% then extend on the first timescale that is the immediate context

% first (milliseconds to seconds): immediate perceptual moment, binding in time and context, being able to make useful predictions at all
%/- discrimation classes of prediction error (Wolpert)
%/- sensory prediction errors (shadmehr), visuomotor adaptation (shadmehr)
%/- cerebellar (shadmehr, krakauer)
%- spatial conflict processing
%- parietal (Savoie, Ridderinkhof); posterior parietal sensory motor integration within the prediction error hierarchy
%- task and temporal structure

% second (minutes to hours): overarching task goal, optimal reinforcement learning, reward expectation
%- focus on frontal (cohen, cavanagh, toellner, zander, Holroyd, Gehrke): acc/PEN/FRN/ERN: we dont hypothesize ACC since we do not provide a reward in the task. There is no evaluation with respect to a task goal

%The classical work of \citet{Holroyd2002} relates the origin of neural responses to \textit{errors} to anterior cortical sources as part of the motor control filtering system which regulates motor behavior based on prediction errors that are detected and communicated by the basal ganglia. Following their work, numerous studies focused on further disentangling the specific violations frontal brain sources correlate with. Their activity is now often related to valence expectation prediction errors based on learning a given higher-order task objective across trials, such as reducing reaction times or moving a cursor with a brain-computer interface \cite{Zander2016, Cavanagh2010a, Li2011}. Frequently, as in \citet{Holroyd2002} original work, reaction time modulations are characterized as a metric for behavioral adaptation \cite{Holroyd2002}.

% third (days to month): transfer learning, skill acquistion etc.

% then end with hypotheses:
%In our task, we did not specifically provide a task obective across trials

% - Here, we investigate exteroceptive prediction errors during reaching movements similar to reaching for a cup. 
%Head-mounted virtual reality in combination with expected haptic feedback renders a \textit{naturalistic} experience. Further, we presume that predicting the hidden causes of the observed outcomes within the action-perception cycle should, hypothetically, depend on how present one feels in the environment. Virtual reality can provide experiences that elicit profound presence experiences \cite{}. 
%We introduced a paradigm to assess whether prediction error-related potentials at fronto-central electrodes can, on average, detect an increase in the level of immersion, operationalized through added feedback channels \cite{Gehrke_2019}. In the paradigm, we rendered a perturbation in a fraction of the trials, distorting predicted visual or visuo-haptic consequences of the motor command \cite{oddball}.
%Here, the goal of haptic devices is to render realistic sensory feedback that mimic the sensory experiences when acting on the real world
%is proprioceptive feedback important in our task: I'd argue against that because I consider vibration on the fingertip to be an exteroceptive sensation and do not consider it relevant where the hand was and how my joints were angled etc. during the mismatch event, therefore i stick to exteroceptive prediction errors only!
% results in light of applied human-computer interaction scenarios where achieving haptic realism is one of the key challenges, specifically in virtual reality development /cite{Gehrke2019}
%From an active inference standpoint, motor commands are explainable as descending proprioceptive prediction errors from the top-level within the hierarchical brain structure \cite{Adams2013}