\subsection{Processing spatio-temporal asynchronies \textcolor{green}{Lukas, Sezen, Klaus, Pedro}}
In contemporary motor control theories, a hierarchy of generative models is assumed in order to explain potential variability between elicited motor commands, the movement they trigger, and changes in body pose and the environment \cites{Wolpert2011, Shadmehr2010}. Assuming the brain entertains models of its environment, such a forward model predicts the sensory consequences of an action given the current context and efference copy of the motor command, meanwhile refining the models of the causes of those sensory consequences \cite{Pearson2011, Friston2010, Friston2016a}. Evidence about the exact nature of the computational implementation in the brain is converging on Bayesian inference computing predictions from a generative model or \textit{prior beliefs}, that is, what is expected to happen, with sensory \textit{evidence}, what is currently happening \cite{Knill2004}. Taken together, within these frameworks, model evidence is scored by an accuracy or surprise metric, for example, prediction errors. Importantly, in order to make predictions in a stable environment, where inference on the latent variables governing sensory events is useful, the brain must be a good model itself of this environment, grounded in causality \cite{Friston2016a}.

The human sensorium has developed through evolution in order to actively sample the environment with regards to sourcing food and reproduction. Individual senses each excel in specific sensory domains with a combination of reliability-weighted sensory cues providing a holistic sample \cite{Fetsch2012, Cao2019}. Such samples gain their perceptual "significance through their meaning about their causes" \cite{Kording2007}. Thus, in order to efficiently infer what causes certain sensory events, an optimally weighted multisensory perceptual experience is preferable with temporal correlations and spatio-temporal congruencies serving as binding cues \cite{Robertson2003}. Recent evidence points towards a parietal-frontal cortical hierarchy binding sensory signals to a common source while segregating those from diverging sources. Unisensory encoding in domain-specific sensory cortices assuming separate source origins evolves via a reliability-weighted fusion in parietal-temporal areas assuming common source origins, ultimately leading to causal inference in frontal and parietal lobes \cite{Cao2019, Rohe2019}.

Previously, we proposed a paradigm with an asynchronous spatio-temporal cue in a fraction (25\%) of trials, prematurely triggering successful object selection in an active object selection task in virtual reality \cite{Gehrke2019}. Successful object selection was indicated by a color-change of the selected object accompanied by a synchronous vibration under the fingertip hinting at haptic rigidity of the object. By maintaining spatio-temporal synchronization on 75\% of the trials, we hypothesized the formation of a predictive model on the causes of sensory events, therefore amplifying a spatio-temporal prediction error in the asynchronous trials. By investigating participants in this unconstrained reaching setting, we extended classical setups enabling ongoing proprioceptive and vestibular sensing while actively sampling peripersonal space. The tight coupling of spatio-temporal asynchronies to self-generated sampling of the environment may specifically trigger processes in relation to bodily self-consciousness as it anchors predictions to the acting self \cite{Blanke2015, Guterstam2015a}. Parts of the parietal cortex are ideally located for early fusion and binding of multisensory cues as well as integrating spatial reference frames anchored to the bodily self \cite{Clark2018, Gramann2009}. Further connections to the hippocampus via retrosplenial cortices facilitate storing of spatial as well as temporal context about action outcomes \cite{Pearson2011}. Our motivation was to investigate how the current sensory context spanning from proprioceptive to visual cues influences processing of spatio-temporal asynchronies eliciting prediction errors on the latent causal structure of the environment.
% now more detail about EEG of parietal cortex during binding mismatch, causal inference in multisensory perception, sense of agency, locating the self...





%%%% writing resources below:
% situate task in general on three timescales
%In order to situate our report within the literature, we first consider how instantaneous prediction errors are influenced at different time and contextual scales. 

%In the long run, the brain is in the game of optimizing the outcome of its motor behaviors. On any given motor task, learning is ultimately driven by optimizing a trade-off between domain-specific and generalizable motor skill.

% triggering prediction errors at different time and contextual scales.
% then extend on the first timescale that is the immediate context

% first (milliseconds to seconds): immediate perceptual moment, binding in time and context, being able to make useful predictions at all
%- discrimation classes of prediction error (Wolpert)
%- sensory prediction errors (shadmehr), visuomotor adaptation (shadmehr)
%- cerebellar (shadmehr, krakauer)
%- spatial conflict processing
%- parietal (Savoie, Ridderinkhof); posterior parietal sensory motor integration within the prediction error hierarchy
%- task and temporal structure

% second (minutes to hours): overarching task goal, optimal reinforcement learning, reward expectation
%- focus on frontal (cohen, cavanagh, toellner, zander, Holroyd, Gehrke): acc/PEN/FRN/ERN: we dont hypothesize ACC since we do not provide a reward in the task. There is no evaluation with respect to a task goal

%The classical work of \citet{Holroyd2002} relates the origin of neural responses to \textit{errors} to anterior cortical sources as part of the motor control filtering system which regulates motor behavior based on prediction errors that are detected and communicated by the basal ganglia. Following their work, numerous studies focused on further disentangling the specific violations frontal brain sources correlate with. Their activity is now often related to valence expectation prediction errors based on learning a given higher-order task objective across trials, such as reducing reaction times or moving a cursor with a brain-computer interface \cite{Zander2016, Cavanagh2010a, Li2011}. Frequently, as in \citet{Holroyd2002} original work, reaction time modulations are characterized as a metric for behavioral adaptation \cite{Holroyd2002}.

% third (days to month): transfer learning, skill acquistion etc.

% then end with hypotheses:
%In our task, we did not specifically provide a task obective across trials

% - Here, we investigate exteroceptive prediction errors during reaching movements similar to reaching for a cup. 
%Head-mounted virtual reality in combination with expected haptic feedback renders a \textit{naturalistic} experience. Further, we presume that predicting the hidden causes of the observed outcomes within the action-perception cycle should, hypothetically, depend on how present one feels in the environment. Virtual reality can provide experiences that elicit profound presence experiences \cite{}. 
%We introduced a paradigm to assess whether prediction error-related potentials at fronto-central electrodes can, on average, detect an increase in the level of immersion, operationalized through added feedback channels \cite{Gehrke_2019}. In the paradigm, we rendered a perturbation in a fraction of the trials, distorting predicted visual or visuo-haptic consequences of the motor command \cite{oddball}.
%Here, the goal of haptic devices is to render realistic sensory feedback that mimic the sensory experiences when acting on the real world
%is proprioceptive feedback important in our task: I'd argue against that because I consider vibration on the fingertip to be an exteroceptive sensation and do not consider it relevant where the hand was and how my joints were angled etc. during the mismatch event, therefore i stick to exteroceptive prediction errors only!
% results in light of applied human-computer interaction scenarios where achieving haptic realism is one of the key challenges, specifically in virtual reality development /cite{Gehrke2019}
%From an active inference standpoint, motor commands are explainable as descending proprioceptive prediction errors from the top-level within the hierarchical brain structure \cite{Adams2013}