\begin{abstract}

Neural interface technology holds significant promise to track user experience implicitly. Today, it finds increasing application in VR/AR as it allows user assessment without breaking the immersive experience. In VR, designing immersion is the key challenge. Unfortunately, the established metric to assess the effectiveness of immersive VR simulations relies on questionnaires. In this work, we present a complimentary metric based on a Brain-Computer Interface. For the metric to be useful beyond prototypical applications, the neural signal employed must be reliable. Hence, it is beneficial to target the signal's cortical origin directly, separating signal from noise. We designed a reach-to-touch paradigm in VR to probe EEG and movement adaptation to visuo-haptic glitches. Our working hypothesis was, that these glitches, or violations of the predicted action outcome, may indicate a disrupted user experience. The classification scheme using trial-to-trial movement adaptation to classify VR glitches did not exceed chance level performance. However, using Prediction Error EEG features, we classified VR glitches with ~77\% accuracy. We localized the EEG sources driving the classification and found midline cingulate and a distributed network of parieto-occipital EEG sources to enable the classification success. Hence, Prediction Error EEG features from these sources reflect violations of user's predictions during interaction with AR/VR, serving as as a robust, targeted, marker for adaptive user interfaces.

% Please include a maximum of seven keywords
\keywords{EEG, Virtual Reality, BCI, Neural Interface Technology, Post-error Slowing, Prediction Error, Predictive Coding}
\end{abstract}