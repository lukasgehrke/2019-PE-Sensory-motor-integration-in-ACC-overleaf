
75 difference ERP pro Bedingung
average correct
average incorrect
IPQ Ergebnisse
Zusammenhang zw. IPQ und difference ERP Amplituden
epoch noisy => 8 Durchgänge, 10% der Trials sind dann gereinigt
motor readiness potential von resting zu starting point
=> starting point ist näher am signal und hat relevanterer Aktivität
den starting point als 0 und dann subtraktive single trial baseline berechnung
anstelle average baseline, was dann als signal sauberer
single trial velocity parameters von Einzel-ERPs
IC clustering nach 2 mal AMICA
=> auf Study level?
=> ACC spezifisch clustern oder andere clusters anschauen
Sensor-based:
FCz velocity ERP vs. velocity
Gewichtungsparameter identifizieren
Source based:
Gewichtungsparameter von sensor-space in source-based anzuschauen
ACC related trials mit bewegungsparameter
Ursprung zu rekonstrieren
=> beide Ursprünge werden zu einem gemeinsamen Modell erarbeitet
Avinash hatte daran gearbeitet: Velocity Parameter in Sydney Datensatz bewerten, noch nicht fertig, frage nach.
velocity, beschleunigung, reaktionszeit
=> höhere varianz mit höhere geschwindigkeit mehr slag im system => erp effect
velocity, beschleunigung, reaktionszeit, erp
=> univariate x 3
=> multivariate alle 3
reaktionszeit => nachgeordnete Fragestellung
3 positionen von cube => nicht genug trials um das als faktor zu untersuchen
vergleich efferent copy und motor feedback => wie lang dieser vergleich zeit in anspruch nimmt, wird prädizieren, ob erp amplitude davon beeinflusst wird.
einfluss auf latenz: nicht klar zu definieren.
error detection klar: ACC ist region of interest
erro detection vs. bewegungsparameter
nachträglich wäre erst interessant was nach Error detection passiert, vermutlich wird das parietal passieren. gibt es Potenzial andere areale, die damit zusammenhängen?
interaktionsanalyse:
höhere precision auch bei höheren geschwindigkeit durch additional sensorischen feedback / bei erhöhter immersion
abhängiger maß: velocity auf erp
mean von beta-erp für visual und visual vibro
=> erp kontrast (permutation)
=> das ergibt dann die Interaktionsvariable
holroyd & thorsten's Paper : region of interest for ACC (taken from https://www.researchgate.net/figure/MNI-coordinates-of-the-five-ACC-sub-regions_fig1_298797902)
keine behavioral measures notwendig. kurz und knappe paper ist besser.
time-frequency analyse: ein-dimensional oder zwei-dimensional?
vorteil unfold toolbox: convolution
=> nur wenn zw. events overlap existiert, dann ergibt es sinn
=> zusätzliche events reinnehmen und dann durch unfold jagen
=> vermutlich keinen größeren Effekt bringen
potential name for paper: mobi using motion parameters to inform brain dynamic analysis
overleaf: link wird geteilt