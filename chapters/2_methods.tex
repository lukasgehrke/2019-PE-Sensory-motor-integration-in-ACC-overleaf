\section{Materials \& Methods}

\begin{comment}

Below is the list of the general idea for the data analysis:
\begin{itemize}
    \item ACC related trials mit bewegungsparameter, error detection klar: ACC ist region of interest, error detection vs. bewegungsparameter
    \item Avinash hatte daran gearbeitet: Velocity Parameter in Sydney Datensatz bewerten, noch nicht fertig, frage nach. velocity, beschleunigung, reaktionszeit
    \item höhere varianz mit höhere geschwindigkeit mehr slag im system -> erp effect velocity, beschleunigung, reaktionszeit
    \item vergleich efferent copy und motor feedback => wie lang dieser vergleich zeit in anspruch nimmt, wird prädizieren, ob erp amplitude davon beeinflusst wird.
    \item 3 positionen von cube -> nicht genug trials um das als faktor zu untersuchen
    \item nachträglich wäre erst interessant was nach Error detection passiert, vermutlich wird das parietal passieren. gibt es Potenzial andere areale, die damit zusammenhängen? interaktionsanalyse: höhere precision auch bei höheren geschwindigkeit durch additional sensorischen feedback / bei erhöhter immersion abhängiger maß: velocity auf erp mean von beta-erp für visual und visual vibro
    \item follow-up analysis: einfluss auf latenz: nicht klar zu definieren., reaktionszeit -> nachgeordnete Fragestellung
\end{itemize}

Below is the raw list of data processing:
\begin{itemize}
    \item ICA and preprocessing according to bemobil pipeline
    \item study level clustering based on ROI located at ACC (holroyd \& thorsten's Paper : region of interest for ACC)
    \item clean mismatch epochs with eeglabs autorej function (8 Durchgänge, 10\% der Trials sind dann gereinigt)
    \item baseline of mismatch ERP: select 300 ms time window after trial is started and participant wait at least 1s for box to appear (includes motor readiness potential von resting zu starting point, starting point ist näher am signal und hat relevanterer Aktivität; den starting point als 0 und dann subtraktive single trial baseline berechnung anstelle average baseline, was dann als signal sauberer waere)
    \item single trial velocity parameters von Einzel-ERPs
\end{itemize}

\end{comment}


\subsection{Participants}
We recruited 20 participants (\todo{gender and age of participants} female), all right-handed. No participant had experienced VR with vibrotactile feedback. Participants received 10 Euro per hour as compensation for their participation. The study design was approved by the local ethics committee and all participants provided written informed consent prior to their participation. 

\subsection{Procedure}
The apparatus and general procedure are explained in detail elsewhere \citep{Gehrke_2019}. Here, we focus on the description of EEG data processing using richer data of more trials and participants. Compared to \citep{Gehrke_2019} we focus solely on two feedback conditions, visual only (V) as well as visual together with vibration at the fingertip (VT). Due to the interest in processing mismatches, we run linear regression analyses only on those trials.

\subsection{EEG data processing}

this can be a general text for the bemobil preprocess function

\subsection{Ethical Standards}
