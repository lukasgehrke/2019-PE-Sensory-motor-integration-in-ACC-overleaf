\subsection{EEG Preprocessing, Independent Component Analysis (ICA) and motion capture preprocessing}
EEG data preprocessing and ICA were performed with Matlab 2017a and 2019b (MATLAB, The MathWorks Inc., Natick, MA, USA), using the EEGLAB toolbox \cite{Delorme2004a} and the custom 'BeMoBIL Pipeline' scripts and functions \footnote{https://github.com/MariusKlug/bemobil-pipeline}. The single subject data was lowpass filtered with 124Hz and subsequently down-sampled to 250Hz. Channels which were contaminated with artifacts were automatically rejected using the PREP pipeline \cite{Bigdely-Shamlo2015} 'FindNoisyChannel' function, which is selecting bad channels by amplitude, the signal to noise ratio and correlation with other channels. Rejected channels were then interpolated while ignoring the EOG channel, and finally re-referenced to average reference (data A). The data was then filtered with a 1 Hz highpassfilter (data B) and a first adaptive mixture independent component analysis, AMICA \cite{Palmer2011}, was used to identify eye related independent components (ICs) which were projected out of the sensor data. For this, the rank was reduced by one for average reference use and further by the number of interpolated channels in the respective data set. To identify eye components IClabel \cite{Pion-Tonachini2019} was used, whereas components exceeding a value of 0.7 for the 'eye' class were defined as eye components. Then, to detect segments of noisy data, an automated time domain cleaning (see \citet{Gramann2018}) was performed on narrowly filtered data from 1 to 40 Hz. The data was therefore first split into 1 second long segments for which the mean absolute amplitude and standard deviation of all channels as well as the Mahalanobis distance of all channel mean amplitudes were calculated. All three methods results were then joined together in order to rank all segments. The 12\% highest ranking noisy segments were selected for rejection and an additional buffer of $\pm 0.49$ sec was added around each segment resulting in about 15\% rejected data for each subject. This data was rejected from data B and a second AMICA was calculated on this time domain cleaned data. A dipole fitting procedure was performed for each spatial filter using the 10-20 standard electrode locations and a boundary element head model (BEM) based on the MNI brain (Montreal Neurological Institute, MNI, Montreal, QC, Canada). The spatial filter information was then copied back to the preprocessed, interpolated and average referenced data set (data A). Ultimately, all ICs with a 'brain' probability smaller than .5 as indicated by ICLabel were projected out of the data resulting in the final dataset of likely brain sources and their projections to the channels investigated in all subsequent analyses. Across the study set, 217 independent components were retained forming a representative sample of about 11.4 (sd = 4.6) components per participant. Motion capture data was filtered with a 6Hz lowpass filter and upsampled to match EEG frequency using MoBILAB routines for concurrent analyses \cite{Ojeda2014}. Subsequently the first derivative was computed and 3D magnitudes were extracted.

\subsection{Event-related Feature Extraction}
Data was epoched -3 to +2 seconds around the 600 \textit{cube selection} events and trials were removed if (a) the reaction time between cube \textit{spawn} and \textit{selection} exceeded two seconds or (b) large voltage fluctuations in the channels were detected via EEGLABs \textit{autorej} function. On average 83.7 (sd = 37.6) trials were removed. For each remaining single trial we considered event-related potentials for channels and independent components (ERP), event-related velocities, i.e. the magnitude of velocities in x, y and z direction (ERV) as well as event-related time-frequency decompositions (event-related spectral perturbation, ERSP). Time-frequency decomposition were computed via the \textit{newtimef} function in EEGLAB for 3 to 80 Hz in logarithmic scale, using a wavelet transformation with 3 cycles for the lowest frequency and a linear increase with frequency of 0.5 cycles. Subsequently, phase was discarded and raw power values were kept for further analysis. Where applicable, grand average ERSPs are computed by first averaging both trial data and baselines across trials in power then dividing trial data by baseline and transforming the outcome to logarithmic scaling ($dB = 10*log_{10}(power)$). ERPs are plotted after bandpass filtering with a low and high cutoff at 0.1 and 15 Hz respectively.

\subsection{Classifier, Classifier Scalp Projections and Localization of Components relevant to Classification}
Following \citet{Zander2016} a regularized linear discriminant analysis classifier was trained per participant with all asynchronous trials constituting class 1 and a random sample of an equal size of synchronous trials labeled class 2. Using Matlab 2014a (MATLAB, The MathWorks Inc., Natick, MA, USA) with the open-source toolbox BCILAB ver. 1.4 the classifier was trained on windowed means as features. Therefore, average amplitudes of all channels and eight sequential 50 ms time windows between 50 and 450 ms following \textit{cube selection} were extracted following resampling to 100 Hz and band-pass filtering from 0.1 to 15 Hz. A mean baseline taken in the 0 to 50 ms window post event was subtracted in order to compensate for event classes occurring at different stages of the ongoing movement. For robust performance estimation, a 5 x 5 nested cross-validation was used to calculate the classifiers reliability. Classification accuracy was statistically evaluated using a two-sample ttest with the mean classifier accuracy per participant across folds and simulated chance level given trial numbers in each class \cite{Muller-Putz2007}. In order to learn what regions of the brain the classifier specifically relied on, we first transformed the LDA filters at each time window to LDA patterns reflecting a mixture of scalp activations with regards to the discriminative source activity. Subsequently, the relevance for classification can be computed using LDA filter weights per time window and the ICA umixing matrix \cites{Haufe2014a, Zander2016}. The equivalent current dipole models of independent components were then weighted by their relevance and ultimately visualized via EEGLAB dipoleDensity plots \cite{Krol2019}. We established a \textit{highest relevance for classification} voxel via visual inspection at $[0, -35, 50]$ (in MNI space, reflecting a source located in or near the superior parietal, BA40) used as a seed region of interest (ROI) for targeted optimization of group-level IC clustering.
% maybe add movie for supplement material
% todo cite dipoledensity
%\footnote{Available at https://sccn.ucsd.edu/wiki/EEGLAB_ Extensions_and_plug-ins}
%and $[20, -65, 30]$ 

% make clear that class size was subsampled

\subsection{Clustering Independent Components for Group-Level Analyses}
To allow for group-level analyses across independent components, we clustered components based on their equivalent dipole locations using a region of interest (ROI) driven repetitive k-means clustering approach \cite{Gramann2018}. ICs were clustered by applying the k-means algorithm with k equals 13, the median number of ICs retained across participants. ICs with a distance of more than three standard deviations from any final centroid mean were considered outliers. After applying desirable weights (number of participants: 2, ICs/participants: -2, spread: -1, RV: -1, distance from ROI: -2, Mahalanobis distance from the median: -1) the final clustering solution contained 21 ICs from 17 participants, that is a ratio of ~1.2 ICs per participant, a normalized spread (mean squared distances from individual dipoles to cluster centroid) of 371.8 $mm^2$, a mean RV of 5.8\%, and a distance of 5.1$mm$ to the ROI. Four participants exhibited two ICs contained in the optimized cluster. We chose to average their activity following the assumptions of our clustering approach.
% no overlapping ICs in clustering twice, one parietal, one visual association
% old solution
% (weight=6), grand-average ERSPs (weight=3), mean log spectra (weight=1), and scalp topography (weight=1),
 %The weighted IC measures were reduced to a 10-dimensional feature vector for clustering via PCA. 