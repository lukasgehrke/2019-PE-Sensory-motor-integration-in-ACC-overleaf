\subsection{Dataset}
\subsubsection{Participants}
20 participants (12 female, mean age = 26.7 (sd = 3.6)) were recruited through an online tool provided by the Department of Psychology and Ergonomics and through local listings. Participants were right-handed, had normal or corrected to normal vision and had not experienced VR with either vibrotactile feedback at the fingertip or any form of force feedback, including EMS. Participants were compensated with 10 Euros per hour or study credits. Participants were informed of the nature of the experiment, recording and anonymization procedures and signed a consent form approved by the local ethics committee of the Department of Psychology and Ergonomics at the TU Berlin. Data of the first subject had to be removed from further analyses due to data recording error. 

% TODO:
% [] add ethics approval credentials

\subsubsection{Recordings: Motion Capture and EEG}
EEG was recorded using 64 active Ag/AgCl electrodes placed according to the extended international 10–20 system \cite{Chatrian1985a}. The electrode at position FP2 was detached from the cap and placed under the left eye to provide additional information about eye movements (EOG). Impedance was kept under 30 \si{\kohm} and the EEG was sampled at 500 Hz and amplified using BrainAmp DC amplifiers (Brainproducts GmbH, Gilching, Germany). Hand and head movements were sampled at 90 Hz when coming out of the HTC Vive processing cascade. Samples were recorded and synchronized using labstreaminglayer \footnote{https://github.com/sccn/labstreaminglayer}.

\subsubsection{Reproducing Results and Data Availability}
Data, experimental protocol, analyses code and earlier publications are accessible from a comprehensive repository hosted at open science foundation (OSF)~\footnote{https://osf.io/x7hnm/}. BIDS formatted data is hosted on openneuro~\cite{} and a full reproduction of the presented results is feasible.

% TODO
% [] add citation to openneuro dataset