\subsection{Multilevel modeling}
\subsubsection{First-level parameter estimation by single-trial regression}
% describe the model and what each parameter is
% - lmfit using matlab
% we did not include any other covariates as their influence was controlled for by a pseudorandomized design, i.e. direction was controlled, sequence? and trial number?

\subsubsection{Robust group-level inference \textcolor{green}{Lukas, Marius, Klaus}}
% describe which effects were analyzed and describe the method
% - robust ttests using 20% trimmed means

Spectra and event-related spectral perturbations (ERSP\citep{Makeig2004}) were analyzed on the group level using robust repeated-measures ANOVA\citep{Pernet2011} to test for the effect of repetition and multiple regressions to assess the influence of each behavioral parameter individually. Initially, we obtained estimates of each subjects response by mass-univariate modeling of each ERSP time-frequency point. We fitted the model $Y = Maze*Repetition + Touch_Duration$ to control for the effect of touch duration. Since allowing continuous interaction with the touchable walls introduced the confound of touches/epochs of different duration. The resulting regression estimates (Betas) were used as each single-subjects summary for group-level inferences.

To characterize the mapping between event-related brain dynamics and the stimulus properties of the time-locked events we employed a hierarchical linear modeling approach on the time-frequency data. First, single-trial regressions weights were estimated for the factors maze and trial run, their interaction, as well as a continuous noise predictor (touch duration “slope rERSP”) for each time-frequency pixel across trials [Rousselet 2008,2009,2010]. We chose a resolution of 96 frequencies  (~3 to ~100 Hz) and 177 timepoints (-300ms to 714ms) around the wall touch event. Since single trials were warped linearly to the same length we corrected for the warping by regressing the touch duration (see above) on each time-frequency pixel across single trials. Before regression, touch duration was centered on 0. Hence, estimates for the categorical predictors are interpreted when touch duration is at its mean value. After adding back the intercept and summing across the corresponding regression terms for each of the 12 factor levels (3 Maze Trials X 4 Mazes), single trial estimates were averaged (in power) for each factor level. Whole epoch average power per factor level was then baseline corrected using a divisive baseline, i.e. mean power values in each frequency across the -300 to -100ms window pre wall touch. Subsequently, power values were transformed to the decibel scale using 10 * log10 (factor level average baseline corrected power).

Using LIMO EEG, subject estimates from the first level were taken as input for group level inference testing [Pernet2011,Rousselet2011,Cohen2011]. To investigate the main effect of both factors, run and maze, we computed 3 (Runs) X 4 (Maze) repeated-measures ANOVAs for each time-frequency pixel for each IC k-means cluster independently. We addressed the multiple comparison problem by correcting the FWER using the threshold-free cluster-enhancement statistic (TFCE, [cite smith and nicols, 2014]). TFCE statistical maps were computed for each of 1000 bootstraps obtained by sampling with replacement across participants. A distribution of maximum TFCE scores was build, keeping the max TFCE score per bootstrapped time-frequency map. The TFCE scores of the true result were then thresholded with the 0.95 percentile (alpha = 0.05) score of the max TFCE distribution.

In order to investigate the relationship between ERSPs and parameters of spatial learning, we build a regression model using group level data as input. Taking factor level data as the dependent variable, we entered both sketchmap scores and exploration duration as continuous predictors, assuming a continuum of a mental spatial representation represented by discrete sketchmap scores. For simplicity, we did not consider any interaction terms with the experimental design (3 maze trials X 4 maze). As above, to correct for multiple comparisons, TFCE scores of bootstrapped samples were used to build a max TFCE distribution to determine the alpha level threshold value.

For visualization purposes, group average ERSPs were obtained by averaging dB time-frequency maps significance masked by threshold surviving TFCE scores.
% end clean up