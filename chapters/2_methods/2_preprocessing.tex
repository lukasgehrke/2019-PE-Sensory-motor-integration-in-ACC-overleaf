%todo
% how many channels were removed on average?
% what where the ICLabel settings for cleaning?
% update with standard text of bemobil pipeline
\subsection{Motion capture and EEG preprocessing \textcolor{green}{Lukas, Marius, Klaus}}

EEG-data was recorded from 64 active electrodes with a sampling rate of 500 Hz (Brainproducts GmbH, Gilching, Germany). Electrodes were mounted on an elastic cap (cite here: EASYCAP, Herrsching, Germany) with placing according to the extended international 10–20 system \cite{chatrian_ten_1985}. The electrode at position FP2 was detached from the cap and placed under the left eye, in order to measure participants Electrooculogram (EOG). Impedances were kept under 30 \si{\kohm}. The data preprocessing and analysis were performed in Matlab 2014a and 2019b  (MATLAB, The MathWorks Inc., Natick, MA, USA), using the EEGLAB toolbox \cite{delorme_eeglab:_2004} the 'BeMoBIL Pipeline' \cite{klug2018bemobil} as well as the BCILAB \cite{} plugin. The single subject data was lowpass filtered with 124Hz and down-sampled to 250Hz. Channels which were contaminated with artifacts were automatically rejected using the PREP pipeline \cite{bigdely-shamlo_prep_2015} 'FindNoisyChannel' function, which is selecting bad channels by amplitude, the signal to noise ratio and correlation with other channels. Rejected channels were then interpolated while ignoring the EOG channel, and finally re-referenced to average reference (data A). The data was then filtered with a 1 Hz highpassfilter (data B), a first adaptive mixture independent component analysis, AMICA \cite{palmer_newton_2008}, was used to identify eye related independent components (ICs) which were projected out of the sensor data (data A). For this, the rank was reduced by one for average reference use and further, the number of interpolated channels in the respective data set. To identify eye components IClabel  \cite{pion2019iclabel} was used, whereas components exceeding a value of 0.7 for the 'eye' class were defined as eye components.
Then, to detect segments of noisy data, an automated time domain cleaning (see \citet{gramann2018heading}) on the time domain was performed on narrowly filtered data from 1 to 40 Hz. The data was therefore first split into 1 second long segments for which the mean absolute amplitude and standard deviation of all channels as well as the Mahalanobis distance of all channel mean amplitudes were calculated. All three methods results were then joined together in order to rank all segments. The 12\% highest ranking noisy segments were selected for rejection and an additional buffer of +/- 0.49 sec was added around each segment resulting in about 15\% rejected data for each subject. This data was rejected from data B and a second AMICA was calculated on this time domain cleaned data. A dipole fitting \cite{oostenveld2002validating} was performed for each spatial filter and the spatial filter information was then copied back to the preprocessed, interpolated and average referenced data set (data A).

%To study event-related potentials, a 0.1 Hz to 40 Hz filter was applied  and eye components as well as line noise components were projected out, again with component labeling using ICLabel (both thresholds set to 0.8). The continuous data streams were then epoched from 1200 ms pre-target stimuls (200ms before reference stimulus) to 1200 ms post-target stimulus presentation. Automatic epoch cleaning was applied using the same procedure as described above, rejecting 10\% of the noisiest epochs.
%All epochs containing mismatch trials, incorrect responses, as well as reaction times exceeding two standard deviations (sd) from participants mean reaction time in the respective difficulty level, were excluded from further analysis on EEG and behavioral level. The same procedures were applied to create longer epochs up to 6000 ms post-target stimulus in order to properly plot the ERP Image \cite{delorme2015grand} sorted by reaction times. (check number of remaining trials etc and add here)

\subsection{EEG source separation and clustering}

%two step IC classification process: 1. using ICLabel removing eyes for classical ERP analyses reproducing previous findings, 2. selecting a template scalp projection from the literature on frontal theta and error monitoring and extracting scalp maps across subjects with high correlation, if more than one scalp map per subject, select one with higher correlation coefficient (using published method corrmap) -> this is one approach to address the clustering problem of independent EEG sources. We chose it in order to best situate our findings within the prevalent literature. Ultimately we fitted a single equivalent dipole to the average scalp map across the study sample to provide a hint at a potential origin of our measured signal.

Artifactual channels were manually identified, removed, and interpolated using spherical interpolation (on average, X channels were interpolated, SD=Y). Subsequently, the data was re-referenced to the average of all channels and a zero-phase Hamming windowed high-pass FIR filter (order 827, pass-band edge 1 Hz) was applied to the data. Concatenated data from all individual explorations was parsed into maximally independent components (IC) using the adaptive mixture of independent component analyzers (AMICA) algorithm with prior principal component analysis (PCA) reduction to the remaining rank after interpolation\citep{Palmer2011}. Subsequently, independent components reflecting eye activity were subtracted from the data in order for an automated cleaning procedure across channels in the time domain not to be affected by eye signals. All Components were automatically matched against the ‘ICLabel’ database and components with a probability >= .XY to reflect eye activity were subtracted\citep{iclabel}. Subsequently, 15\% of the noisiest time segments were determined by jointly ranking the amplitude mean, standard deviation, and mahalanobis distance across channels of consecutive one-second epochs\citep{cleaning_fh2018}. Subsequently, artifactual time windows were rejected from the concatenated data prior to eye component removal and a second ICA was computed on this cleaned data containing eye movement signals. Hence in a first step eye movement components were rejected to subsequently find noisy segments in the data not related to eye movements.

For each IC, an equivalent dipole model was computed as implemented by DIPFIT routines in EEGLAB. Using the 10-20 standard electrode locations, a a boundary element head model (BEM) based on the MNI brain (Montreal Neurological Institute, MNI, Montreal, QC, Canada) was used with DIPFIT routines. We refer to the approximated spatial origin of an IC as “in or near” a specified location.

% todo
% ICLabel settings selecting components
To allow for group-level comparisons of EEG data at the source level (ICs), components with a probability of XY using ICLabel were selected and subsequently clustered based on their equivalent dipole locations (weight=6), grand-average ERSPs (weight=3), mean log spectra (weight=1), and scalp topography (weight=1), using a region of interest (ROI) driven repetitive k-means clustering approach\citep{cleaning_FH2018}. The weighted IC measures were summed and compressed using PCA, resulting in a 10-dimensional feature vector for clustering. ICs were clustered by applying the k-means algorithm with k equaling 70\% of the mean number of ICs retained across subjects. We chose to use fewer clusters than ICs per participant because of our assumption that, although statistically independent per time point, there may be more than one IC per participant that is similar in function and location. ICs with a distance of more than three standard deviations from any final centroid mean were considered outliers.

% todo: update with final clustering solution properties
To ensure replicability of the clustering, we employed the same clustering approach reported in\cite{cleaning_FH2018}. After applying desirable weights (number of participants: 3, ICs/participants: -1, spread: -1, RV: -1, distance from ROI: -2, Mahalanobis distance from the median: -1) the final clustering solution contained the ICs from 25 participants, a ratio of 1.13 ICs per participant (two participants with two ICs each), a spread of 296, a mean RV of 10.3\%, and a distance of 10.2 units in the Talairach space for the cluster of primary interest.