\subsection{Processing}

\subsubsection{Brain activity: EEG Preprocessing, Independent Component Analysis (ICA)}
EEG data preprocessing and ICA were performed in Matlab 2019b (MATLAB, The MathWorks Inc., Natick, MA, USA), using EEGLAB toolbox \cite{Delorme2004a} and custom 'BeMoBIL Pipeline' scripts and functions \footnote{https://github.com/BeMoBIL/bemobil-pipeline}. Single subject data were lowpass filtered with 124Hz and subsequently down-sampled to 250Hz. Channels which were contaminated with artifacts were automatically rejected using the PREP pipeline \cite{Bigdely-Shamlo2015} 'FindNoisyChannel' function, which is selecting bad channels by amplitude, the signal to noise ratio and correlation with other channels. Rejected channels were then interpolated while ignoring the EOG channel, and finally re-referenced to average reference (data A). The data was then filtered with a 1 Hz highpassfilter (data B) and a first adaptive mixture independent component analysis, AMICA \cite{Palmer2011}, was used to identify eye related independent components (ICs) which were projected out of the sensor data. For this, the rank was reduced by one for average reference use and further by the number of interpolated channels in the respective data set. To identify eye components, IClabel \cite{Pion-Tonachini2019} was used, whereas components exceeding a value of 0.7 for the 'eye' class were defined as eye components. Then, to detect segments of noisy data, an automated time domain cleaning (see \citet{Gramann2018}) was performed on narrowly filtered data from 1 to 40 Hz. The data was therefore first split into 1 second long segments for which the mean absolute amplitude and standard deviation of all channels as well as the Mahalanobis distance of all channel mean amplitudes were calculated. All three methods results were then joined together in order to rank all segments. The 12\% highest ranking noisy segments were selected for rejection and an additional buffer of $\pm 0.49$ sec was added around each segment resulting in about 15\% rejected data for each subject. This data was rejected from data B and a second AMICA was calculated on this time domain cleaned data. A dipole fitting procedure was performed for each spatial filter using the 10-20 standard electrode locations and a boundary element head model (BEM) based on the MNI brain (Montreal Neurological Institute, MNI, Montreal, QC, Canada). The spatial filter information was then copied back to the preprocessed, interpolated and average referenced data set (data A). 

Ultimately, all ICs with a probability smaller than .7 as indicated by the ICLabel 'brain' class were projected out of the data resulting in the final dataset of very likely brain sources and their projections to the channels investigated in all subsequent analyses. Across the study set, 271 independent components were retained forming a representative sample of about 14.3 (sd = 5.0) components per participant. 

\subsubsection{Behavior: Motion Capture}
Motion capture data was filtered with a 6Hz lowpass filter and upsampled to match EEG frequency using MoBILAB routines for concurrent analyses \cite{Ojeda2014}. Subsequently the first derivative was computed and velocity was extracted. 

We assessed post-error adaptation as a function of change in 'action time': the time elapsed between the start of the reaching movement following object spawn and the end of that movement. The reach onset was detected by stepping back from the peak velocity of the reach and selecting the first sample where the velocity fell below 0.05 m/s. The end of the reach was determined as the first sign change of the change in z-direction, the primary reach direction, following the start of the reach. Trial-to-trial changes in action time were computed to assess the effectiveness of the experimental manipulation~\cite{Dutilh2012}. As such, action time was independent from the too-early appearance of the mismatch feedback. 

We modeled the occurrence of mismatch trials with a linear mixed effects model employing the trial-to-trial change rate in action time. The model $mismatch ~ change_in_action_time + (1 | participant_id)$ was fit using Matlab's 'fitglme' function. The predictive accuracy of the model was assessed using 10-fold cross-validation. To assess the models effectiveness, a two-sample ttest was computed using each fold's accuracy and the simulated chance level considering the the classes sample size in each fold~\cite{Muller-Putz2007}.

\subsubsection{EEG Classifier, Classifier Scalp Projections and Localization of Components relevant to Classification}
Following \citet{Zander2016} a regularized linear discriminant analysis classifier was trained per participant with all mismatch trials constituting class 1 and a random sample of an equal size of match trials labeled class 2. Using the open-source toolbox BCILAB ver. 1.4 the classifier was trained on windowed means as features. First, EEG data were re-sampled to 100 Hz and band-pass filtered from 0.1 to 15 Hz. Average amplitudes of all channels in eight sequential 50 ms time windows between 50 and 450 ms after the cube was touched were extracted, the windowed means feature vectors. A mean baseline taken in the 0 to 50 ms window post event was subtracted in order to compensate for event classes, match and mismatch, occurring at different stages of the ongoing movement. For robust performance estimation, a 5 x 5 nested cross-validation was used to calculate the classifiers reliability. Classification accuracy was statistically evaluated using a two-sample ttest with the mean classifier accuracy per participant across folds and simulated chance level given trial numbers in each class \cite{Muller-Putz2007}.

In order to learn what regions of the brain the classifier specifically relied on, we first transformed the LDA filters at each time window to LDA patterns reflecting a mixture of scalp activations with regards to the discriminative source activity. Subsequently, the relevance for classification can be computed using LDA filter weights per time window and the ICA umixing matrix \cites{Haufe2014a, Zander2016}. The equivalent current dipole models of independent components were then weighted by their relevance and ultimately visualized via EEGLAB dipoleDensity plots \cite{Krol2019}. 

\subsection{Mismatch processing in midcingulate independent components}
We established a \textit{highest relevance for classification} voxel via visual inspection at $[0, -35, 50]$ (in MNI space, reflecting a source located in or near the superior parietal, BA40) used as a seed region of interest (ROI) for targeted optimization of group-level IC clustering.

% TODO adapt new solutions ROI here

\subsubsection{Clustering Independent Components}
To allow for group-level analyses across independent components, we clustered components based on their equivalent dipole locations using a region of interest (ROI) driven repetitive k-means clustering approach \cite{Gramann2018}. ICs were clustered by applying the k-means algorithm with k equals 14, the median number of ICs retained across participants. ICs with a distance of more than three standard deviations from any final centroid mean were considered outliers. After applying desirable weights (number of participants: 2, ICs/participants: -2, spread: -1, RV: -1, distance from ROI: -2, Mahalanobis distance from the median: -1) the optimal target cluster 

% TODO adapt clustering solution here

solution contained 21 ICs from 17 participants, that is a ratio of ~1.2 ICs per participant, a normalized spread (mean squared distances from individual dipoles to cluster centroid) of 371.8 $mm^2$, a mean RV of 5.8\%, and a distance of 5.1$mm$ to the ROI. Four participants exhibited two ICs contained in the optimized cluster. We chose to average their activity following the assumptions of our clustering approach.

\subsubsection{Modeling Event-related Spectral Perturbations}
To keep all task events from cube spawn to touch, data were epoched -3 to +2 seconds around the 600 cube touches and trials were removed if (a) the reaction time between cube \textit{spawn} and touch exceeded two seconds or (b) large voltage fluctuations in the channels were detected via EEGLABs \textit{autorej} function with default settings.

% TODO add epoch removal info and modeling, mcc approach

On average 83.7 (sd = 37.6) trials were removed. For the clean single trials we computed event-related time-frequency decompositions (event-related spectral perturbation, ERSP) of each independent component time course. Time-frequency decomposition were computed via the \textit{newtimef} function in EEGLAB for 3 to 80 Hz in logarithmic scale, using a wavelet transformation with 3 cycles for the lowest frequency and a linear increase with frequency of 0.5 cycles. Subsequently, phase was discarded, the output squared and resulting raw power values kept for further analysis.

To obtain first-level, per participant, summaries of ERSP, mass-univariate multiple regression was computed across all mismatch trials. Therefore, a linear model was estimated at each time-frequency pixel of participants independent component(s) present in the ROI cluster. The linear model was defined as \textit{tf\textunderscore pixels = intercept + match\textunderscore mismatch * haptic\textunderscore feedback + baseline}. Match\textunderscore mismatch differentiated match trials from mismatch trials and haptic\textunderscore feedback vibrotactile from those missing the added immersive channel. To further infer whether components of the event-related response could be explained by baseline activity, average power per frequency bin in the -200 to 0 ms window preceding cube \textit{spawns} was entered as a predictor as well.

% TODO sample size in ROI cluster(s)

For inference, a permutation t-test using EEGlab's \textit{statcond} function with 1000 permutations was employed on the regression betas. Due to the small sample size in the ROI cluster (N = 14) we chose the permutation-t approach over a cluster statistic with constrained resampling for multiple comparison correction~\cite{Pernet2015}. First, for each permutation the betas map of t-scores was transformed to a map of tfce-scores. Next, for each tfce-map the maximum was extracted and the true tfce-map was thresholded at the 95th percentile of the max-tfce distribution from the 1000 permutations. 





%%%%% resources
% \subsubsection{Modeling Event-related Spectral Perturbations in "Midcingulate" Independent Components}
% % Linear modeling of Clusterwise Event-related Spectral Perturbation
% % specify the model and briefly state to what end it was designed, which questioned it was supposed to answer 

% \subsubsection{Single-Trial Multiple Regression, Group-level Statistics and Multiple Comparison Correction}
% In order to describe task execution relevant components of the event-related spectral perturbations of the entire epoch, we conducted a t-test of the grand average cluster ERSP, baseline corrected using grand average divisive baseline. Employing a permutation t-test using the function \textit{statcond} with 1000 permutations we controlled for multiple comparisons. Due to the small sample size (N = 17) we chose the permutation-t approach over a cluster statistic with constrained resampling \cite{Pernet2015}. This procedure was also used to assess inference of betas obtained from mass-univariate single-trial regressions as described below. With respect to the grand-average, slight shifts preceding the event of interest due to randomized pre-trial intervals were ignored and averaged across. This smearing in time was accepted since time-frequency resolution similarly reduces temporal accuracy. Since single-trial analysis only focused on the interval succeeding the event of interest, that is the feedback associated with touching the cube or, in case of mismatch trials, the premature feedback, randomized pre-trial intervals did not impact these analyses.
% % todo % citation statcond (fieldtrip and eeglab)
% %In order to correct for multiple comparison we transformed the resulting map of t-scores to a map of tfce-scores and thresholded the tfce-map at the 95th percentile of the max-tfce distribution of 600 bootstrapped t-tests using as implemented in LIMO EEG. Due to the low number of 17 participants in each analysed cluster, we constrained bootstraps to contain at least 14 unique participants to overcome conservative multiple comparison correction \cite{Pernet2015}.

% % better split grand-average and single-trial analyses
% \subsubsection{Single-trial regression to disentangle spatio-temporal binding prediction errors}
% To obtain first-level, per participant, summaries of event-related spectral perturbations, mass-univariate multiple regression was computed across all mismatch trials. Therefore, a linear model was estimated at each time-frequency pixel of participants independent component(s) present in the ROI cluster. The linear model was defined as \textit{tf\textunderscore pixels = intercept + hand\textunderscore velocity * haptic\textunderscore feedback + RT + Baseline}. Instantaneous hand velocity was extracted at the moment of object selection. Haptic feedback differentiated trials including vibrotactile from those missing the added immersive channel. In order to estimate the interaction term between hand velocity and haptics, both predictors were normalized prior to model fitting. Reaction time was operationalized as the time elapsed from the object spawning on the table to the object being selected. To further infer whether components of the event-related response could be explained by baseline activity, average power per frequency bin in the -200 to 0 ms window preceding cube \textit{spawns} was entered as a predictor as well.
% %optional if results are promising -> \subsection{correlation post-error slowing and theta}

% \subsubsection{Velocity perturbations following spatio-temporal binding prediction errors}
% To investigate whether ongoing motor behavior following spatio-temporal binding manipulations changed as a function of time or in response to haptic feedback a mass-univariate multiple regression was computed across all mismatch trials. The linear model \textit{velocity = intercept + trial\textunderscore number * haptic\textunderscore feedback} was fit at all time points addressing whether (a) trials were comparable across time and (b) whether rendering the surface by means of tactile feedback impacted movement execution potentially hinting at perceived object rigidity. At the group-level, a robust one-sample t-test was computed for betas of trial number and haptic feedback. To assess whether initiation and motor execution slowed down following spatio-temporal binding perturbations per participant betas were obtained by fitting the linear model \textit{velocity = intercept + following\textunderscore asynchrony * following\textunderscore haptics} at all time points across the epoch followed by group-level inference as described above.

% no overlapping ICs in clustering twice, one parietal, one visual association
% old solution
% (weight=6), grand-average ERSPs (weight=3), mean log spectra (weight=1), and scalp topography (weight=1),
 %The weighted IC measures were reduced to a 10-dimensional feature vector for clustering via PCA. 
 
 % maybe add movie for supplement material
% todo cite dipoledensity
%\footnote{Available at https://sccn.ucsd.edu/wiki/EEGLAB_ Extensions_and_plug-ins}
%and $[20, -65, 30]$ 
% make clear that class size was subsampled

% On average 83.7 (sd = 37.6) trials were removed. For each remaining single trial we considered event-related potentials for channels and independent components (ERP), event-related velocities, i.e. the magnitude of velocities in x, y and z direction (ERV) as well as event-related time-frequency decompositions (event-related spectral perturbation, ERSP). Time-frequency decomposition were computed via the \textit{newtimef} function in EEGLAB for 3 to 80 Hz in logarithmic scale, using a wavelet transformation with 3 cycles for the lowest frequency and a linear increase with frequency of 0.5 cycles. Subsequently, phase was discarded and raw power values were kept for further analysis. Where applicable, grand average ERSPs are computed by first averaging both trial data and baselines across trials in power then dividing trial data by baseline and transforming the outcome to logarithmic scaling ($dB = 10*log_{10}(power)$). ERPs are plotted after bandpass filtering with a low and high cutoff at 0.1 and 15 Hz respectively.
