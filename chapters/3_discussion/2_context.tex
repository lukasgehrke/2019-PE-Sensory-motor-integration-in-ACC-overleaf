\subsection{Neural signatures of multisensory context during spatio-temporal asynchrony}
% summary of discussion part: dissociating alpha and theta band activity
% - alpha: baseline and reaction time
% - theta: haptics and interaction velXhap / theta - (what about potential theta-gamma coupling?)
% - why velocity not impacting ersp: little variation, highly automated task other mobi challenges
% - reference back to intro part: %Prominently, theta- and alpha-band activity, as well as their finer granularities, have been implicated in cognitive processing. \citet{VanRullen2016} argued for the notion of \textit{perceptual cycles} alluding to these frequencies as the impulse generators of perception. \citet{Rohe2019} implicate prestimulus alpha frequencies in \textit{rhythmic perception} modulating the tendency whether following spatio-temporal binding challenges are more likely to resolve in binding or segregation. The extensive research on theta frequencies, with a frequent localized to frontal areas, have concluded a role in cognitive control \cite{Cavanagh2014}. Recently, \citet{Duprez2020} have demonstrated theta frequencies as a messenger directing behavioral adaptation through cognitive control. Following a cognitive demand for adaption, theta motivates rich behavioral policy changes exceeding simple reaction time measures \cite{Cooper2019}.