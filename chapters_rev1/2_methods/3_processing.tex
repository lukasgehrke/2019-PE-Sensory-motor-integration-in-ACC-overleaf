\subsection{Processing}

% \textcolor{red}{\st{\subsubsection{Predicting VR glitches using Behavioral Adaptation}}}
\subsubsection{Behavioral Adaptation Following VR Glitches}

Motion capture data was filtered with a 6Hz low-pass filter and re-sampled to match the EEG sample rate using MoBILAB routines for concurrent analyses~\cite{Ojeda2014-ev}. Subsequently the first derivative was computed and velocity was extracted. 

We computed `tap time', the time elapsed between the start of the reaching movement following object spawn and the end of that movement, using the hand velocity time series. The reach onset was detected on the hand velocity time series by moving backwards from the velocity peak of the reach movement and selecting the first sample where the velocity fell below 0.05 m/s. The end of the reach was determined as the first sign reversal of the movement change in z-direction, the primary reach direction, following the start of the reach. \textcolor{red}{\st{As such, tap time was independent from the premature appearance of the mismatch feedback.}}
\textcolor{n}{As such, tap time was detected on the continuous time series and not on the experimental event. This would have been problematic since the premature appearance of the mismatch feedback event would have artificially created an effect.}

To assess behavioral adaptation, \textcolor{n}{we modeled the \textit{rate of change} in `tap time' with a linear model. To this end, we computed the difference in `tap time' between subsequent trials and report this}\textcolor{red}{\st{the tap time of each mismatch trial was subtracted from the tap time of the subsequent match trial to obtain the}} \textit{rate of change} as a response to the experimental manipulation~\cite{Dutilh2012-ps}. We reported tap time instead of reaction time since participants were not primed, nor did they receive any reward for fast and accurate trial completion. \textcolor{red}{\st{Likewise, missing any framing of the task for speed and accuracy, we did not report a correlation analyses linking tap time to EEG signals.}} \textcolor{n}{The model \textit{`change in tap time $\sim$ trial change'} was fitted using Matlab's `fitlm' function and assessed using `anova'. Trial change was entered as a categorical predictor reflecting whether the current trial change was match to mismatch, mismatch to match or match to match. Since the number of consecutive match trials was pseudo-randomized between 1 and 5 we decided to exclude trials where a mismatch trial occurred again after the first subsequent match trial. These trials corresponded to both the mismatch to match and match to mismatch trial change category. This resulted in the removal of 30 Trials per participant for the behavioral analysis.}

%`Sequence' described the number of consecutive match trials preceding a mismatch trial. The current `sequence' value at each mismatch trial was applied to both the match to mismatch as well as the subsequent mismatch to match trial. 

\textcolor{red}{\st{We modeled the occurrence of match/mismatch trials with a generalized linear (mixed effects) model employing the \textit{rate of change} in tap time as a predictor. The model \textit{`Feedback $\sim$ change in tap time + (1 | Participant ID)'} was fit using Matlab's `fitglme' function. To assess the model, we calculated a variance-ratio test using Matlab's `compare' function. In order to comply with the ERP classification scheme outlined below, we randomly sub-sampled trials from the match class to match the trial count in the mismatch class. The predictive accuracy of the model was assessed using a within-subject 5-fold cross-validation. In the within-subject case, the model is identical to a generalized linear model \textit{`Feedback $\sim$ change in tap time}. To assess the models effectiveness, a two-sample T-test was computed using participants average accuracy across folds and the simulated chance level under consideration of the classes sample size}}~\cite{Muller-Putz2007-oc}.

\subsubsection{Brain activity: EEG Preprocessing, Independent Component Analysis (ICA)}
EEG data preprocessing and ICA were performed in Matlab 2019b (MATLAB, The MathWorks Inc., Natick, MA, USA), using the EEGLAB toolbox~\cite{Delorme2004-sn} and custom `BeMoBIL Pipeline' scripts and functions. To detect bad channels for rejection, the `FindNoisyChannel' function was used, which is selecting bad channels by amplitude, the signal to noise ratio and correlation with other channels ~\cite{Bigdely-Shamlo2015-ds}. Rejected channels were then interpolated while ignoring the EOG channel, and finally re-referenced to average reference (data A). The data was then filtered with a 1 Hz high-pass filter (data B) and a first adaptive mixture independent component analysis, AMICA~\cite{Palmer2011-zs}, was used to identify eye related independent components (ICs) which were projected out of the sensor data. For this, the rank was reduced by one for the use of an average reference and further by the number of interpolated channels in the respective data set. To identify eye components, IClabel~\cite{Pion-Tonachini2019-fy} was used, whereas components exceeding a value of 0.7 for the 'eye' class were defined as eye components. Then, to detect segments of noisy data, an automated time domain cleaning (see~\cite{gramann2021human}) was performed on narrowly filtered data from 1 to 40 Hz. The data was therefore first split into 1 second long segments for which the mean absolute amplitude and standard deviation of all channels as well as the Mahalanobis distance of all channel mean amplitudes were calculated. All three methods results were then joined together in order to rank all segments. The 12\% highest ranking noisy segments were selected for rejection and an additional buffer of $\pm 0.49$ sec was added around each segment resulting in about 15\% rejected data for each subject. This data was rejected from data B and a second AMICA was calculated on this time domain cleaned data. A dipole fitting procedure was performed for each spatial filter using the 10-20 standard electrode locations and a boundary element head model (BEM) based on the MNI brain (Montreal Neurological Institute, MNI, Montreal, QC, Canada). The spatial filter information was then copied back to the preprocessed, interpolated and average referenced data set (\textcolor{n}{see description of} data A \textcolor{n}{above}).

\textcolor{n}{To obtain indices of clean tap epochs, we leveraged EEGLAB's `pop\textunderscore autorej' function to remove epochs exhibiting large amplitude fluctuations. We used the functions default settings and entered epochs from -3 to 2 seconds surrounding the tap events. On average, 80.7 epochs were rejected (SD = 32.6) amounting to $\sim$13\% of the data.}

Ultimately, all ICs with a probability smaller than .7 as indicated by the ICLabel `brain' class were projected out of the data \textcolor{red}{\st{resulting in the final dataset of very likely brain sources and their projections to the channels}}\textcolor{n}{This resulted in the final dataset including only very likely brain sources and their projections to the channels}. Across the study set, 271 independent components were retained forming a representative sample of about 14.3 (SD = 5.0) components per participant.\textcolor{red}{\st{investigated in all subsequent analyses.}} \textcolor{n}{All subsequent EEG analyses were based on these data.} 

\subsubsection{EEG Classifier, Classifier Scalp Projections and Localization of Components relevant to Classification}

In the current work, we present a processing pipeline with slight updates as compared to our \textcolor{red}{\st{original}} \textcolor{n}{earlier} work~\cite{Gehrke2019-og}. To reproduce our previous findings, we report a permutation t-test of the ERP at electrode FCz. Activity at electrode FCz in the time window from 150 to 200 ms post mismatch event featured prominently in our earlier analysis and is frequently considered for MMN paradigms investigating ERPs at the scalp level, for modeling evidence see~\cite{Lieder2013-dl, Lieder2013-os}. \textcolor{n}{For completeness, we report all electrodes that exhibited an amplitude difference at 200 ms post tap event. To this end, we computed a t-test of the amplitudes at 200ms post tap event. To correct for multiple comparisons, the false discovery rate (\textit{fdr}) was computed with alpha = 05~\cite{Benjamini1995-cw}. Channels whose p-value exceeded the \textit{fdr} were plotted, see figure~\ref{erp}.}

For classification \textcolor{n}{of single-trial ERPs}, we followed the approach introduced by \cite{Zander2016-ed}. A regularized linear discriminant analysis classifier was trained per participant with all mismatch trials constituting class 1 and a random sample of an equal \textcolor{red}{\st{size}}\textcolor{n}{number} of match trials labeled class 2. Using the open-source toolbox BCILAB ver. 1.4, the classifier was trained on windowed means as features. First, EEG data were re-sampled to 100 Hz and band-pass filtered from 0.1 to 15 Hz. Average amplitudes of all channels in eight sequential 50 ms time windows between 0 and 400 ms after the cube was tapped were extracted as the windowed means feature vectors. A mean baseline taken in the -50 to 0 ms window was subtracted in order to compensate for event classes, match and mismatch, occurring at different stages of the ongoing movement. For robust performance estimation, a 5 x 5 nested cross-validation was used to calculate the shrinkage regularization parameter and assess the classifiers performance.

Classification accuracy was statistically evaluated using a two-sample T-test with the mean classifier accuracy per participant across folds and simulated chance level given trial numbers in each class \cite{Muller-Putz2007-oc}.

In order to learn what regions of the brain the classifier specifically relied on, we first transformed the LDA filters at each time window to LDA patterns reflecting a mixture of scalp activations with regards to the discriminative source activity \cite{Haufe2014-do}. \textcolor{n}{Subsequently, each independent component's relevance for classification was computed as the dot product of the LDA patterns per time window and the ICA unmixing matrix filter weights} \cite{Zander2016-ed}. The equivalent current dipole models of independent components were then weighted by their relevance and ultimately visualized via EEGLAB `dipoleDensity' plots \cite{Krol2018-cw}. The Harvard-Oxford atlas was consulted to extract cortical labels of regions of interest \cite{Makris2006-kp}.




%%%%% resources [do not review]
% old:
% Using Matlab's `fitlme' function, the model \textit{`action time \textasciitilde  Feedback * Haptics + (1 | Participant Id)'} was fit to assess the alteration of 'action time through the experimental manipulation. Similarly, the model \textit{`change in action time \textasciitilde  Feedback * Haptics + (1 | Participant Id)'} was fit to assess post-error slowing. `Feedback' differentiated match from mismatch trials and `Haptics' vibrotactile from those missing the added immersive channel. Next, the occurrence of match/mismatch trials was modeled with a generalized linear mixed effects model employing the \textit{rate of change} in action time. The predictive accuracy of the model was assessed using 10-fold cross-validation \textit{across} participants. To assess the models effectiveness, a two-sample T-test was computed using each fold's accuracy and the simulated chance level considering the classes sample size in each fold~\cite{Muller-Putz2007-oc}.

% \subsubsection{Modeling Event-related Spectral Perturbations in "Midcingulate" Independent Components}
% % Linear modeling of Clusterwise Event-related Spectral Perturbation
% % specify the model and briefly state to what end it was designed, which questioned it was supposed to answer 

% \subsubsection{Single-Trial Multiple Regression, Group-level Statistics and Multiple Comparison Correction}
% In order to describe task execution relevant components of the event-related spectral perturbations of the entire epoch, we conducted a t-test of the grand average cluster ERSP, baseline corrected using grand average divisive baseline. Employing a permutation t-test using the function \textit{statcond} with 1000 permutations we controlled for multiple comparisons. Due to the small sample size (N = 17) we chose the permutation-t approach over a cluster statistic with constrained resampling \cite{Pernet2015}. This procedure was also used to assess inference of betas obtained from mass-univariate single-trial regressions as described below. With respect to the grand-average, slight shifts preceding the event of interest due to randomized pre-trial intervals were ignored and averaged across. This smearing in time was accepted since time-frequency resolution similarly reduces temporal accuracy. Since single-trial analysis only focused on the interval succeeding the event of interest, that is the feedback associated with touching the cube or, in case of mismatch trials, the premature feedback, randomized pre-trial intervals did not impact these analyses.
% % todo % citation statcond (fieldtrip and eeglab)
% %In order to correct for multiple comparison we transformed the resulting map of t-scores to a map of tfce-scores and thresholded the tfce-map at the 95th percentile of the max-tfce distribution of 600 bootstrapped t-tests using as implemented in LIMO EEG. Due to the low number of 17 participants in each analysed cluster, we constrained bootstraps to contain at least 14 unique participants to overcome conservative multiple comparison correction \cite{Pernet2015}.

% % better split grand-average and single-trial analyses
% \subsubsection{Single-trial regression to disentangle spatio-temporal binding prediction errors}
% To obtain first-level, per participant, summaries of event-related spectral perturbations, mass-univariate multiple regression was computed across all mismatch trials. Therefore, a linear model was estimated at each time-frequency pixel of participants independent component(s) present in the ROI cluster. The linear model was defined as \textit{tf\textunderscore pixels = intercept + hand\textunderscore velocity * haptic\textunderscore feedback + RT + Baseline}. Instantaneous hand velocity was extracted at the moment of object selection. Haptic feedback differentiated trials including vibrotactile from those missing the added immersive channel. In order to estimate the interaction term between hand velocity and haptics, both predictors were normalized prior to model fitting. Reaction time was operationalized as the time elapsed from the object spawning on the table to the object being selected. To further infer whether components of the event-related response could be explained by baseline activity, average power per frequency bin in the -200 to 0 ms window preceding cube \textit{spawns} was entered as a predictor as well.
% %optional if results are promising -> \subsection{correlation post-error slowing and theta}

% \subsubsection{Velocity perturbations following spatio-temporal binding prediction errors}
% To investigate whether ongoing motor behavior following spatio-temporal binding manipulations changed as a function of time or in response to haptic feedback a mass-univariate multiple regression was computed across all mismatch trials. The linear model \textit{velocity = intercept + trial\textunderscore number * haptic\textunderscore feedback} was fit at all time points addressing whether (a) trials were comparable across time and (b) whether rendering the surface by means of tactile feedback impacted movement execution potentially hinting at perceived object rigidity. At the group-level, a robust one-sample t-test was computed for betas of trial number and haptic feedback. To assess whether initiation and motor execution slowed down following spatio-temporal binding perturbations per participant betas were obtained by fitting the linear model \textit{velocity = intercept + following\textunderscore asynchrony * following\textunderscore haptics} at all time points across the epoch followed by group-level inference as described above.

% no overlapping ICs in clustering twice, one parietal, one visual association
% old solution
% (weight=6), grand-average ERSPs (weight=3), mean log spectra (weight=1), and scalp topography (weight=1),
 %The weighted IC measures were reduced to a 10-dimensional feature vector for clustering via PCA. 
 
 % maybe add movie for supplement material
% todo cite dipoledensity
%\footnote{Available at https://sccn.ucsd.edu/wiki/EEGLAB_ Extensions_and_plug-ins}
%and $[20, -65, 30]$ 
% make clear that class size was subsampled

% On average 83.7 (sd = 37.6) trials were removed. For each remaining single trial we considered event-related potentials for channels and independent components (ERP), event-related velocities, i.e. the magnitude of velocities in x, y and z direction (ERV) as well as event-related time-frequency decompositions (event-related spectral perturbation, ERSP). Time-frequency decomposition were computed via the \textit{newtimef} function in EEGLAB for 3 to 80 Hz in logarithmic scale, using a wavelet transformation with 3 cycles for the lowest frequency and a linear increase with frequency of 0.5 cycles. Subsequently, phase was discarded and raw power values were kept for further analysis. Where applicable, grand average ERSPs are computed by first averaging both trial data and baselines across trials in power then dividing trial data by baseline and transforming the outcome to logarithmic scaling ($dB = 10*log_{10}(power)$). ERPs are plotted after bandpass filtering with a low and high cutoff at 0.1 and 15 Hz respectively.
