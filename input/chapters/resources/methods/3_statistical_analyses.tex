\subsection{Single-Trial Multiple Regression, Group-level Statistics and Multiple Comparison Correction}
In order to describe task execution relevant components of the event-related spectral perturbations of the entire epoch, we conducted a t-test of the grand average cluster ERSP, baseline corrected using grand average divisive baseline. Employing a permutation t-test using the function \textit{statcond} with 1000 permutations we controlled for multiple comparisons. Due to the small sample size (N = 17) we chose the permutation-t approach over a cluster statistic with constrained resampling \cite{Pernet2015}. This procedure was also used to assess inference of betas obtained from mass-univariate single-trial regressions as described below. With respect to the grand-average, slight shifts preceding the event of interest due to randomized pre-trial intervals were ignored and averaged across. This smearing in time was accepted since time-frequency resolution similarly reduces temporal accuracy. Since single-trial analysis only focused on the interval succeeding the event of interest, that is the feedback associated with touching the cube or, in case of asynchronous trials, the premature feedback, randomized pre-trial intervals did not impact these analyses.
% todo % citation statcond (fieldtrip and eeglab)
%In order to correct for multiple comparison we transformed the resulting map of t-scores to a map of tfce-scores and thresholded the tfce-map at the 95th percentile of the max-tfce distribution of 600 bootstrapped t-tests using as implemented in LIMO EEG. Due to the low number of 17 participants in each analysed cluster, we constrained bootstraps to contain at least 14 unique participants to overcome conservative multiple comparison correction \cite{Pernet2015}.

\subsubsection{Velocity perturbations following spatio-temporal binding prediction errors}
To investigate whether ongoing motor behavior following spatio-temporal binding manipulations changed as a function of time or in response to haptic feedback a mass-univariate multiple regression was computed across all asynchronous trials. The linear model \textit{velocity = intercept + trial\textunderscore number * haptic\textunderscore feedback} was fit at all time points addressing whether (a) trials were comparable across time and (b) whether rendering the surface by means of tactile feedback impacted movement execution potentially hinting at perceived object rigidity. At the group-level, a robust one-sample t-test was computed for betas of trial number and haptic feedback. To assess whether initiation and motor execution slowed down following spatio-temporal binding perturbations per participant betas were obtained by fitting the linear model \textit{velocity = intercept + following\textunderscore asynchrony * following\textunderscore haptics} at all time points across the epoch followed by group-level inference as described above.

% better split grand-average and single-trial analyses
\subsubsection{Single-trial regression to disentangle spatio-temporal binding prediction errors}
To obtain first-level, per participant, summaries of event-related spectral perturbations, mass-univariate multiple regression was computed across all asynchronous trials. Therefore, a linear model was estimated at each time-frequency pixel of participants independent component(s) present in the ROI cluster. The linear model was defined as \textit{tf\textunderscore pixels = intercept + hand\textunderscore velocity * haptic\textunderscore feedback + RT + Baseline}. Instantaneous hand velocity was extracted at the moment of object selection. Haptic feedback differentiated trials including vibrotactile from those missing the added immersive channel. In order to estimate the interaction term between hand velocity and haptics, both predictors were normalized prior to model fitting. Reaction time was operationalized as the time elapsed from the object spawning on the table to the object being selected. To further infer whether components of the event-related response could be explained by baseline activity, average power per frequency bin in the -200 to 0 ms window preceding cube \textit{spawns} was entered as a predictor as well.
%optional if results are promising -> \subsection{correlation post-error slowing and theta}