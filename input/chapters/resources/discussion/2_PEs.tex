% filling the gap detail #2: More precisely, resolving prediction error signals in \textit{natural} interaction with virtual worlds
\subsection{Prediction Error ERPs in \textit{Natural} Interaction with Virtual worlds}
% TODO:
% [] already go into affordances here
% [] align with rich literature on PEs, some arguments on why ERP
% - see Cohen 2014 muscle twitches: In our final set of analyses,we examined the ERPs—the time-domain EMG onset-locked EEG potential. Thiswas done mainly to replicate previous findings concerning the relationship between the ERP and partial errors.
% - moving towards an applicable metric to detect things online, hence must be computationally inexpensive, therefore ERPs
% - use ERP section as exemplary for understanding results
% - In haptically richer worlds, processing gets more accurate and hence amplifies the error signal originating in or near anterior cingulate cortex (ACC). Moving fast and experiencing richer haptic feedback impact error processing
% - discuss with spatial conflict processing: Savoie, & Simon Task results: cohen, cavanagh, toellner
% - reference to self and body ownership, spatial computations between egocentric and allocentric? cite Ehrsson, Slater, Gonzales-Franco (uncanny valley of haptics)

Coherent multisensory integration yields meaningful perceptual experiences and is central to adaptive behavior because it allows us to perceive a world of coherent perceptual entities. Minimising prediction error, the objective function of predictive processing, has been extensively studied with regards to perception in motor control. In neuroscientific experiments, engaged viewing as reflected in saccadic eye movements is a frequently studied observable manifestation of how the brain directs motor output during active sampling of the environment. However, environmental affordances surpass the visual domain. Particularly in humans and apes, a proclivity to use both hands to act on the environment has emerged and is greatly trusted upon. In fact, many tasks can be completed without concurrently consulting the visual domain, such as typewriting or even reaching for a cup; these typically rely heavily on the tactile and proprioceptive sense and as such are denoted as eyes-free interactions.

% - discuss with spatial conflict processing: Savoie, & Simon Task results: cohen, cavanagh, toellner
% - reference to self and body ownership, spatial computations between egocentric and allocentric? cite Ehrsson, Slater, Gonzales-Franco (uncanny valley of haptics)
% \subsection{Self-anchoring spatial reference frames}
% - describe necessity to understand where body parts are located in peripersonal space
% - anchoring the body part only works in reference to other objects (allocentric), what is the egocentric perspective?
% - link back to introduction about: %Parietal areas, as a location of arising alpha rhythms have been frequently linked to anchoring the bodily self during action-perception being ideally located for integration of spatial reference frames \cite{Blanke2015, Guterstam2015a, Halgren2019}. Further, connections to the hippocampus via retrosplenial cortices implicate a role in spatio-temporal context situating and reproduction \cite{Pearson2011, Clark2018, Gramann2009, Friston2016a}.

\subsubsection{Localizing Prediction Errors' EEG Source Origin}

