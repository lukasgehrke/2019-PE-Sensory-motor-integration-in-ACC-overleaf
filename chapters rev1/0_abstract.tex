\begin{abstract}

Neural interfaces hold significant promise to implicitly track user experience. Their application in VR/AR simulations is especially favorable as it allows user assessment without breaking the immersive experience. In VR, designing immersion is one key challenge. Subjective questionnaires are the established metrics to assess the effectiveness of immersive VR simulations. However, administering such questionnaires requires breaking the immersive experience they are supposed to assess. In this work, we present a complimentary metric based on a Brain-Computer Interface. For the metric to be robust, the neural signal employed must be reliable. Hence, it is beneficial to target the neural signal's cortical origin directly, efficiently separating signal from noise. To test this new complementary metric, we designed a reach-to-tap paradigm in VR to probe EEG and movement adaptation to visuo-haptic glitches. Our working hypothesis was, that these glitches, or violations of the predicted action outcome, may indicate a disrupted user experience. The classification scheme using trial-to-trial movement adaptation to classify VR glitches did not exceed chance level performance. However, using prediction error negativity features, we classified VR glitches with ~77\% accuracy. We localized the EEG sources driving the classification and found midline cingulate EEG sources and a distributed network of parieto-occipital EEG sources to enable the classification success. Hence, prediction error signatures from these sources reflect violations of user's predictions during interaction with AR/VR, promising a robust and targeted marker for adaptive user interfaces.

% Please include a maximum of seven keywords
% \keywords{EEG, Virtual Reality, BCI, Neural Interface Technology, Post-error Slowing, Prediction Error, Predictive Coding}
\end{abstract}